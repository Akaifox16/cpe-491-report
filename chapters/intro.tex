\chapter{\ifenglish Introduction \else บทนำ\fi}
\section{\ifenglish Project rationale\else ที่มาของโครงงาน\fi}

ระบบจองหอพักนักศึกษามหาวิทยาลัยเชียงใหม่ประสบปัญหาหลายประการ ที่อาจทำให้นักศึกษายากที่จะจองหอพัก 
และหาห้องพักที่เหมาะสม ปัญหาหนึ่งคือความไม่เพียงพอของข้อมูลเกี่ยวกับผู้เข้าพักร่วม 
นักศึกษาที่ไม่มีเพื่อนที่จะอยู่ร่วมกันต้องหาผู้เข้าพักร่วมโดยสุ่ม ซึ่งอาจทำให้เกิดปัญหาความขัดแย้ง 
และยากที่จะประสบความสัมพันธ์ที่ดีกัน ปัญหาความขัดแย้งนี้สามารถเกิดขึ้นได้จากวิถีชีวิต กิจวัตรประจำวัน 
ตารางเวลา หรือแม้แต่บุคคลิกภาพ และนิสัยส่วนตัว ที่ไม่เข้ากันในการเป็มผู้ร่วมอาศัย
อย่างไรก็ตามผู้ดูแลระบบนั้นอนุญาตให้ผู้ใช้งานทุกคนในการแลกห้อง หรือผู้ร่วมอาศัยในช่วงระยะเวลาหนึ่ง 
ซึ่งสามารถช่วยบรรเทาปัญหาดังกล่าวได้ แต่จะเป็นการเพิ่มภาระในงานของผู้ดูแล

ประกอบกับการที่ระบบนั้นประสบปัญหาเรื้อรังจาก การแข่งขันที่สูงของระบบการจอง นักศึกษาถูกบีบบังคับให้รีบตัดสินใจในการจองห้องพัก
ซึ่งนำพาไปสู่การตัดสินใจเลือก และมีความเป็นไปได้ที่จะนำไปสู่ปัญหาการทะเลาะวิวาทมากขึ้น ด้วยที่ระบบนั้นมีการแข่งขันที่สูงเป็นธรรมชาติอยู่แล้วนั้น
ยังสามารถที่จะป้องกันไม่ให้นักศึกษาที่เป็นเพื่อนกันมาก่อน ไม่มีโอกาสได้พักอาศัยร่วมกัน ส่งผลให้ยิ่งทวีความตึงเครียด
และความยากลำบากในการพักอาศัย

นอกจากนี้ความต้องการที่จะพักอาศัยสูงก่อให้เกิดปัญหา เว็บแอพพลิเคชันไม่สามารถให้บริการได้ 
ผู้ใช้ต้องประสบปัญหาอย่างยากลำบากในการเข้าใช้งาน จนสิ้นสุดการทำงานของแอพพลิเคชัน เนื่องจากปัญหาทางเทคนิค 
และผู้ใช้ที่มากเกินความสามารถของเซิร์ฟเวอร์ ปัญหาเหล่านี้สามารถก่อความรำคาญ และความยากลำบากให้กับผู้ใช้ ซ
ึ่งต้องแข่งขันกันในการจองห้องพักที่มีอยู่อย่างจำกัด ในการเลือกที่พักอาศัยที่ต้องการ

เพื่อแก้ปัญหาดังกล่าว โครงงานนี้จึงได้เสนอแอพพลิเคชันใหม่ ที่อนุญาตให้ผู้ใช้ระบุห้อง 
และเพื่อนร่วมห้องตามความต้องการ ซึ่งระยะเวลาที่ระบบเปิดให้ใช้บริการนั้นจะยาวนานขึ้น 
หลังจากนั้นระบบจะทำการปิดระบบ และจับคู่ให้ผู้ใช้ทุกคนแบบอัตโนมัติ โดยอ้างอิงจากความคล้ายคลึงกันของความต้องการของแต่ละฝ่าย 
ด้วยการที่แอพพลิเคชันให้ข้อมูลเกี่ยวกับรูมเมทที่จะได้มากขึ้น และอนุญาตให้ผู้ใช้ระบุความต้องการของตนเอง 
จะทำให้ง่ายต่อระบบที่จะจับคู่เพื่อนร่วมห้องที่เหมาะสม และสามารถหลีกเลี่ยงปัญหาความขัดแย้งได้ 
อีกด้วยว่าการขยายช่วงเวลาที่เปิดให้ใช้งานระบบยาวขึ้น ผู้ใช้จะมีเวลามากขึ้นในการตัดสินใจ จึงสามารถลดระดับการแข่งขัน 
และเพิ่มโอกาสในการหาคู่ที่เหมาะสม อย่างไรก็ตามวิธีการนี้ยังไม่สามารถระบุวิธีการแก้ปัญหาที่เว็บแอพพลิเคชันไม่สามารถให้บริการได้
แต่การลดระดับการแข่งขันลงได้นั้น ก็มีแนวโน้มที่จะช่วยบรรเทาปัญหาดังกล่าวได้ 
ในท้ายนี้วิธีการนี้นั้นจะสามารถพัฒนาประสบการณ์การจองหอพัก และเลือกผู้ร่วมอาศัยของนักศึกษามหาวิทยาลัยเชียงใหม่ให้ดีขึ้นได้อย่างมีนัยสำคัญ
% ในปัจจุบันระบบจองหอพักนักศึกษามหาวิทยาลัยเชียงใหม่ ขาดการอำนวยข้อมูลคุณลักษณะที่เพียงพอของ 
% เพื่อนร่วมห้องที่นักศึกษาจะต้องพักอาศัยด้วย ผู้ใช้ที่ไม่ได้มีเพื่อนที่จะพักอาศัยด้วยแต่แรกจึงต้องสุ่มหาเพื่อนร่วมห้อง
% จึงอาจจะทำให้เกิดปัญหาทะเลาะกันเนื่องจากมีความเป็นไปได้ที่จะได้เพื่อนร่วมห้องที่เข้ากันยาก 
% ประกอบกับผู้ใช้มีโอกาสน้อยมากที่จะได้ใช้เวลาในการคิดตัดสินใจ เนื่องจากว่าต้องแข่งขันในการจองหอพักที่ต้องการ
% ซึ่งจะยิ่งทำให้ปัญหาดังกล่าวข้างต้นแย่ลงกว่าเดิม เพราะกลุ่มผู้ใช้ที่มีเพื่อนที่รู้จักกันมาก่อนอาจจะไม่ได้อยู่ด้วยกัน
% และปัญหาสุดท้ายเป็นผลมาจาก ปัญหาการแข่งขันที่สูงของผู้เข้าใช้งาน ทำให้เกิดปัญหาเว็บไม่สามารถให้บริการได้ตลอดเวลา
% และเรื้อรังเป็นเวลาหลายปี ซึ่งจะยิ่งทวีความรุนแรงของ 2 ปัญหาข้างต้น

% ทางผู้จัดทำจึงมุ่งที่จะแก้ปัญหาดังกล่าวโดยจะนำเสนอเว็บแอพพลิเคชันจองหอพัก ที่จะไม่

% old
% หอพักนักศึกษามหาวิทยาลัยเชียงใหม่ เป็นหอพักที่นักศึกษาชั้นปีที่ 1 ส่วนมากให้ความสนใจ เนื่องจากมีราคาถูก และตั้งอยู่ในเขตมหาวิทยาลัยที่มีโครงสร้างพื้นฐานมากมาย
% ระบบการจองหอพักในปัจจุบันนั้น ประกอบด้วยขั้นตอนทั้งหมดดังนี้
% \begin{enumerate}
%     \item ลงทะเบียนสร้างบัญชีผู้ใช้
%     \item ลงชื่อเข้าใช้งาน
%     \item เลือกห้องพักที่ต้องการพักอาศัย
%     \item เลือกห้องพักที่ต้องการพักอาศัย
%     \item ยืนยันการลงทะเบียน
% \end{enumerate}
% ในแต่ละขั้นตอน มีการส่งคำร้อง (request)ไปยังเครื่องเซิฟเวอร์เพื่อเรียกหน้าเว็บไซต์แต่ละหน้า 
% ทำให้ระบบการจองหอพักไม่สามารถรองรับผู้ใช้งานจำนวนมากได้ในช่วงเวลาหนึ่งๆ กล่าวคือ เมื่อมีจำนวนผู้ใช้งานเป็นจำนวนมาก จะทำให้ระบบไม่ตอบสนองต่อ requests ของผู้ใช้
% นอกจากนี้ เว็บไซต์ยังต้องประมวลผลการลงทะเบียนตลอดเวลาหากห้องที่นักศึกษาต้องการเลือกนั้นไม่ว่างในหอหนึ่งๆ 
% หรือหอที่ต้องการเลือกนั้นเต็มแล้ว นักศึกษาจำเป็นต้องย้อนกลับไปยังหน้าก่อนหน้า ซึ่งเป็นการส่ง request ใหม่ไปยังเซิฟเวอร์เช่นกัน 
% อีกปัญหาหนึ่งที่เกิดขึ้นในระบบจองหอพักปัจจุบัน คือการแข่งขันที่สูง เนื่องจากระบบจะจำกัดห้องให้จองได้ในแต่ละรอบการรับนักศึกษา ทำให้มีผู้ผ่านการคัดเลือกในรอบนั้นๆ ที่ไม่สามารถจองห้องได้ทัน
% และเนื่องจากการแข่งขันที่สูง ส่งผลให้นักศึกษาต้องรีบลงทะเบียน ซึ่งอาจจะทำให้ไม่ได้อยู่หอหรือห้องเดียวกันกับเพื่อนตามความต้องการ ทำให้เพื่อนร่วมห้องที่ได้ในท้ายที่สุดนั้นมาจากการสุ่ม ซึ่งอาจจะทำให้เกิดปัญหาในการอยู่ร่วมกันในภายหลัง
% % นำไปสู่การทะเลาะกัน หรืออยู่ด้วยกันแบบอึดอัดใจกัน
% % \CIreply{ต้นตอปํญหาคืออะไรกันแน่ ที่เขียนมาค่อนข้างจะ claim เยอะ แต่ยังไม่ค่อยเห็นเหตุผลสนับสนุน  ลองดูว่าปัญหาที่แท้จริงคืออะไร และระบบที่มีมันทำให้ปัญหาที่มีอยู่แล้วนั้นแย่ลง หรือเป็นตัวสร้างปัญหาตั้งแต่แรก จะได้แก้ปัญหาได้ตรงจุด}

% ทางผู้จัดทำจึงเลือกที่จะแก้ปัญหาการแข่งขันที่สูงของระบบจองหอพักที่มีในปัจจุบัน เพื่อให้ทุกคนมีสิทธิ์ในการจองหอพักอย่างเท่าเทียมกัน
% โดยการเปลี่ยนแปลงความสามารถจากการระบุห้องหรือหอที่ต้องการได้ทันที เป็นการบอกความต้องการคุณลักษณะของหอพักและห้องพักแทน
% แล้วระบบใหม่จะพยายามเลือกหอและห้องพักให้ตรงตามความต้องการมากที่สุด อีกทั้งจะเพิ่มความสามารถในการจับคู่เพื่อนร่วมห้องให้เป็นไปตามความต้องการของผู้ใช้ ทำให้ผู้ใช้ที่อาจจะไม่มีเพื่อนที่อยู่ด้วยสามารถเลือกเพื่อนร่วมห้องที่เข้ากันได้
% และสามารถลดปัญหาการขัดแย้งกันของเพื่อนร่วมห้อง เนื่องจากเพื่อนร่วมห้องที่ได้จากการจับคู่มีความเข้ากันที่มากขึ้น
% ในท้ายที่สุด ระบบจับคู่ห้องพักและเพื่อนร่วมห้องนี้ จะแก้ปัญหาเซิร์ฟเวอร์ไม่ตอบสนอง ด้วยการย้ายกระบวนการประมวลผลการจองหอพักและการจับคู่ไปเป็นช่วงหลังจากปิดระบบลงทะเบียนแล้ว
% ซึ้งเป็นการลดการประมวลผลของเซิร์ฟเวอร์ในวันลงทะเบียน

% ทางผู้จัดทำจึงได้คิดวิธีการแก้ปัญหาตัวระบบเก่าโดยการลดจำนวน request ที่ส่งมายัง server เพื่อแก้ปัญหา 
% server รองรับ request จำนวนมากไม่ได้ ซึ่งระบบที่จะพัฒนาขึ้นใหม่นั้นจะไม่ประมวลผลระบบ ณ ขณะที่เปิดให้จอง
% แต่จะรอให้ระบบปิดก่อนจึงจะประมวลผล และให้ผู้ใช้กรอกแบบสอบถามเพื่อนำมาใช้ในการเลือกเพื่อนร่วมห้องที่เหมาะสมที่สุด
% \CIreply{ขายระบบมากเกินไป ควรขายเป้าหมายก่อน}

\section{\ifenglish Objectives\else วัตถุประสงค์ของโครงงาน\fi}
\begin{enumerate}
    \item ลดจำนวนผู้ใช้ที่ต้องการย้ายห้องหลังทำการจองแล้ว
    \item เพื่อทดแทนระบบเดิมที่ใช้ในปัจจุบัน
    \item เพื่อลดภาระงานของผู้ดูแลในการดูแลระบบ
          % \item พัฒนาเว็บแอพพลิเคชันตาม\hyperref[subsec:uxlaws]{หลักการออกแบบ UX}
          % \item ลดกระบวนการการคำนวณของเซิร์ฟเวอร์ในวันที่ระบบเปิดให้ลงทะเบียน
          % \item จับคู่รูมเมท, หอพัก และห้องพักไม่ให้เกิด \hyperref[sec:rmp]{rogue couple}
\end{enumerate}

\section{\ifenglish Project scope\else ขอบเขตของโครงงาน\fi}
\subsection{\ifenglish Hardware scope\else ขอบเขตด้านฮาร์ดแวร์\fi}
\begin{enumerate}
    \item เครื่องเซิร์ฟเวอร์ที่ติดตั้ง Docker engine
\end{enumerate}
\subsection{\ifenglish System scope\else ขอบเขตของระบบ\fi}
\begin{enumerate}
    \item ระบบจะรองรับเฉพาะการจองของหอพักนักศึกษามหาวิทยาลัยเชียงใหม่
    \item การจับคู่จะพิจารณาจากคุณสมบัติที่ระบบมีให้เท่านั้น
    \item ผู้ใช้จะไม่สามารถเลือกเฉพาะเจาะจงห้องพัก หรือ หอพักได้โดยตรง
\end{enumerate}
% ]

\section{\ifenglish Expected outcomes\else ประโยชน์ที่ได้รับ\fi}
\begin{enumerate}
    \item ระบบถูกนำไปใช้งานได้จริง
    \item ผู้ใช้งานขอเปลี่ยนห้องน้อยลงหลังจากระบบประมวลผลเสร็จสิ้น
    \item ผู้ดูแลไม่ต้องกลับมาจัดการหรือแก้ไขข้อมูลในระบบบ่อยๆ
    \item ได้แอพพลิเคชันที่มีการออกแบบที่ทันสมัย
    \item ได้แอพพลิเคชันที่ทำงานขัดข้องน้อยลง
          % \item ผู้ใช้เรียนรู้วิธีการใช้งานระบบลงทะเบียนได้เร็วขึ้นกว่าระบบที่มีอยู่เดิม
          % \item สังคมหอพักน่าอยู่ยิ่งขึ้น
          % \item ชื่อเสียงด้านงานพัฒนาคุณภาพชีวิตนักศึกษาของมหาวิทยาลัยเป็นไปในทางที่ดีมากขึ้น
\end{enumerate}

\section{\ifenglish Technology and tools\else เทคโนโลยีและเครื่องมือที่ใช้\fi}

\subsection{\ifenglish Hardware technology\else เทคโนโลยีด้านฮาร์ดแวร์\fi}
\begin{enumerate}
    \item Docker~\cite{dke} -- เป็น container runtime engine ที่ช่วยสร้างและรัน containers จาก Docker images
\end{enumerate}

\subsection{\ifenglish Software technology\else เทคโนโลยีด้านซอฟต์แวร์\fi}
\begin{enumerate}
    \item TypeScript~\cite{typescript} -- ภาษาคอมพิวเตอร์ที่พัฒนาต่อยอดมาจากภาษา JavaScript
          ที่เน้นให้สามารถหาจุดที่จะทำให้เกิดข้อผิดพลาดได้ก่อนที่จะทำการรันแอพพลิเคชัน
    \item React~\cite{react} -- เป็น library ที่ใช้ในการพัฒนา front-end ของเว็บแอพพลิเคชัน
    \item Next.js\cite{nextjs} -- เป็น React Framework ที่ช่วยให้สามารถพัฒนาเว็บแอพพลิเคชันได้ง่ายยิ่งขึ้น
          % \item Gin Gonic~\cite{gingonic} -- เป็น Go Framework ที่ช่วยในการพัฒนา back-end ของเว็บแอพพลิเคชัน
    \item Go~\cite{golang} -- เป็นภาษาคอมพิวเตอร์ที่เข้าใจได้ง่าย และความสามารถที่เด่นในด้านการทำ concurrency
          % \item MySQL~\cite{mysql} -- เป็น relational database management system ที่มีความสามารถในการจัดการ transactions 
          %     และมีความนิยมในการใช้เป็นอย่างมาก
    \item PostgreSQL
          % \item GraphQL~\cite{graphql} --  เป็น query language application programming interface (API) ที่มีลักษณะการใช้งานคล้ายคลึงกับการเขียน query เพื่อดึงข้อมูลจาก MySQL
          % ทำให้ง่ายต่อการนำไปใช้ในการส่งข้อมูลและทำความเข้าใจ
    \item RESTful API
\end{enumerate}
% \CIreply{add reference and brief description to each tool; perhaps further separate into frontend/backend/etc.}

\section{\ifenglish Project plan\else แผนการดำเนินงาน\fi}
\begin{plan}{12}{2021}{2}{2023}
    \planitem{12}{2021}{1}{2022}{ศึกษาค้นคว้าอัลกอริทึม และ งานที่คล้ายคลึงกัน}
    \planitem{1}{2022}{1}{2022}{สอบถามข้อมูลจาก สำนักงานหอพัก}
    \planitem{1}{2022}{2}{2022}{รวบรวมข้อมูล สำหรับการทดสอบระบบ}
    \planitem{1}{2022}{2}{2022}{เลือกเครื่องมือในการพัฒนาระบบ}
    \planitem{2}{2022}{2}{2022}{ศึกษาเรียนรู้เกี่ยวกับเทคโนโลยี และ
        เครื่องมือที่ใช้พัฒนาระบบ}
    \planitem{3}{2022}{5}{2022}{ออกแบบเว็บแอพพลิเคชัน}
    \planitem{10}{2022}{12}{2022}{พัฒนาระบบฐานข้อมูล}
    \planitem{10}{2022}{1}{2023}{พัฒนาเว็บแอพพลิเคชัน}
    \planitem{1}{2023}{2}{2023}{ทดสอบระบบ}
    \planitem{2}{2023}{2}{2023}{เขียนรายงานสรุปผลของการพัฒนาระบบ}
\end{plan}

% \section{\ifenglish Roles and responsibilities\else บทบาทและความรับผิดชอบ\fi}
% อธิบายว่าในการทำงาน นศ. มีการกำหนดบทบาทและแบ่งหน้าที่งานอย่างไรในการทำงาน จำเป็นต้องใช้ความรู้ใดในการทำงานบ้าง

\section{\ifenglish%
      Impacts of this project on society, health, safety, legal, and cultural issues
  \else%
      ผลกระทบด้านสังคม สุขภาพ ความปลอดภัย กฎหมาย และวัฒนธรรม
  \fi}
% แนวทางและโยชน์ในการประยุกต์ใช้งานโครงงานกับงานในด้านอื่นๆ 
% รวมถึงผลกระทบในด้านสังคมและสิ่งแวดล้อมจากการใช้ความรู้ทางวิศวกรรมที่ได้

โครงงานนี้มีส่วนช่วยในการลดเหตุทะเลาะวิวาทของผู้พักอาศัยตลอดช่วงเวลาที่พัก
เนื่องจากได้พักอาศัยกับเพื่อนร่วมห้องที่ต้องการ ยิ่งไปกว่านั้นแอพพลิเคชันยังสามารถช่วยส่งเสริมด้านสังคม 
และวัฒนธรรม ด้วยเหตุที่ผู้ใช้นั้นจะได้เพื่อนร่วมห้องมาจากการสุ่ม ทำให้มีโอกาสได้เจอเพื่อนใหม่ที่มีความชอบคล้ายๆกัน 
ได้สร้างเครือข่ายที่จะคอยส่งเสริมกันในอนาคต นอกจากนี้แอพพลิเคชันยังช่วยให้ผู้ดูแลไม่ต้องแบ่งจำนวนห้อง 
หรือรอบประมวลผลหลายๆ รอบเพื่อรองรับกับระบบ TCAS ในการรับนักศึกษาอีกต่อไป
ผู้ดูแลจะไม่ต้องวิตกกังวลว่าระบบจะล่มทุกครั้งที่เปิดให้ประมวลผลหรือไม่ จึงทำให้ช่วยส่งเสริมสุขภาพร่างกาย 
และสุขภาพจิตของผู้ดูแลให้ดียิ่งขึ้น


% ในการทำโครงงานนี้ คาดว่านักศึกษาจะมีคุณภาพชีวิตที่ดียิ่งขึ้น ได้รู้จักเพื่อนใหม่ อยู่ร่วมกันอย่างมีความสุข
% นอกจากนี้อาจจะทำให้เกิดกิจกรรมใหม่ๆ ขึ้นในหอพัก เนื่องจากคนที่มีความสนใจที่คล้ายๆ กันได้มีโอกาสมาเจอกันมากยิ่งขึ้น
% ยิ่งไปกว่านั้น ทางมหาวิทยาลัยก็จะได้รับชื่อเสียงเพิ่มมากขึ้น เนื่องจากมีระบบการจัดการที่ช่วยให้นักศึกษามีความพึงพอใจในการพักอาศัยภายในหอพักของมหาวิทยาลัย
