ั\chapter{\ifenglish Introduction\else บทนำ\fi}

\section{\ifenglish Project rationale\else ที่มาของโครงงาน\fi}

หอพักในนักศึกษา เป็นหอพักที่นักศึกษาชั้นปีที่ 1 ส่วนมากให้ความสนใจ เนื่องจากราคาที่ถูก 
อีกทั้งตั้งอยู่ในเขตมหาวิทยาลัยที่มีโครงสร้างพื้นฐานมากมายแต่เนื่องจากระบบการลงทะเบียน
มีปัญหาในการรองรับผู้ใช้งานจำนวนมากไม่ได้ เมื่อมีจำนวนผู้ใช้งานเป็นจำนวนมากจะทำให้ระบบ
ไม่ตอบสนองแล้วไม่มี package ส่งกลับมาให้ผู้ใช้งาน ประกอบกับขั้นตอนการลงทะเบียนที่มีหลายขั้น
ตอนคือ 1 ลงชื่อเข้าใช้งาน 2 เลือกหอพักที่ต้องการพักอาศัย 3 เลือกห้องพักที่ต้องการพักอาศัย 
4 ยืนยันการลงทะเบียน ซึ่งในแต่ละขั้นตอนต้องส่ง request เรียกหน้าเว็บไซต์ในทุกขั้นตอน 
อีกทั้งเว็บไซต์ยังประมวลผลการลงทะเบียนตลอดเวลา หากห้องที่เลือก หรือ หอที่เลือกเต็มแล้วต้อง
ย้อนกลับไปหน้าก่อนหน้า ซึ่งเป็นการส่ง request ใหม่ไปยังเซิฟเวอร์เช่นกัน ส่งผลให้นักศึกษาต้องรีบ
ลงทะเบียนให้ได้อยู่ในหอที่ต้องการ หลังจากนั้นจึงไปแลกห้องให้ได้อยู่กับเพื่อนที่ต้องการ ซึ่งอาจจะทำให้
เพื่อนไม่ได้อยู่หอเดียวกัน หรือห้องเดียวกัน ทำให้เพื่อนร่วมห้องที่มี มาจากการสุ่มซึ่งอาจจะทำให้ไม่เข้ากัน
นำไปสู่การทะเลาะกัน หรืออยู่ด้วยกันแบบอึดอัดใจกัน

ทางผู้จัดทำจึงได้คิดวิธีการแก้ปัญหาตัวระบบเก่าโดยการลดจำนวน request ที่ส่งมายัง server เพื่อแก้ปัญหา 
server รองรับ request จำนวนมากไม่ได้ ซึ่งระบบที่จะพัฒนาขึ้นใหม่นั้นจะไม่ประมวลผลระบบ ณ ขณะที่เปิดให้จอง
แต่จะรอให้ระบบปิดก่อนจึงจะประมวลผล และให้ผู้ใช้กรอกแบบสอบถามเพื่อนำมาใช้ในการเลือกเพื่อนร่วมห้องที่เหมาะสมที่สุด

\section{\ifenglish Objectives\else วัตถุประสงค์ของโครงงาน\fi}
\begin{enumerate}
    \item จับคู่เพื่อนร่วมห้องให้ดีที่สุด
    \item จัดการเลือกหอผู้ใช้ ให้ตรงตามความต้องการมากที่สุด
    \item ทำให้ระบบประมวลผลน้อยลง ณ วันที่เปิดระบบ
\end{enumerate}

\section{\ifenglish Project scope\else ขอบเขตของโครงงาน\fi}
\begin{enumerate}
    \item เพื่อนร่วมห้อง, หอพัก และห้องพักจะถูกจับคู่ตามตัวเลือกที่ผู้ใช้เลือก
    โดยให้ไม่มีคู่ใดที่ก่อให้เกิดปัญหา(rouge couple) อยู่เลย
    \item ออกแบบ UX/UI ให้ผู้ใช้ไม่อึดอัดขณะกรอกแบบสอบถาม
    \item 
\end{enumerate}
\subsection{\ifenglish Hardware scope\else ขอบเขตด้านฮาร์ดแวร์\fi}
\begin{enumerate}
    \item เครื่องคอมพิวเตอร์ที่สามารถเชื่อมต่ออินเตอร์เน็ตได้
    \item เครื่องคอมพิวเตอร์ที่สามารถจัดเก็บข้อมูลการลงทะเบียนทั้งหมดได้
\end{enumerate}

\subsection{\ifenglish Software scope\else ขอบเขตด้านซอฟต์แวร์\fi}
\begin{enumerate}
    \item 
\end{enumerate}

\section{\ifenglish Expected outcomes\else ประโยชน์ที่ได้รับ\fi}
\begin{enumerate}
    \item ผู้ใช้ได้รับประสบการณ์ในการใช้งานที่ดีมากขึ้น
    \item นักศึกษาอยู่ร่วมกันอย่างมีความสุข ลดปัญหาการทะเลาะวิวาท และทำให้สังคมหอพักดียิ่งขึ้น
    \item ตัวผู้ศึกษาได้เรียนรู้การออกแบบ UX/UI ที่ดี
\end{enumerate}

\section{\ifenglish Technology and tools\else เทคโนโลยีและเครื่องมือที่ใช้\fi}

\subsection{\ifenglish Hardware technology\else เทคโนโลยีด้านฮาร์ดแวร์\fi}

\subsection{\ifenglish Software technology\else เทคโนโลยีด้านซอฟต์แวร์\fi}

\section{\ifenglish Project plan\else แผนการดำเนินงาน\fi}

\begin{plan}{6}{2020}{2}{2021}
    \planitem{7}{2020}{8}{2020}{ศึกษาค้นคว้า}
    \planitem{8}{2020}{1}{2021}{ชิล}
    \planitem{2}{2021}{2}{2021}{เผา}
    \planitem{12}{2019}{1}{2022}{ทดสอบ}
\end{plan}

\section{\ifenglish Roles and responsibilities\else บทบาทและความรับผิดชอบ\fi}
อธิบายว่าในการทำงาน นศ. มีการกำหนดบทบาทและแบ่งหน้าที่งานอย่างไรในการทำงาน จำเป็นต้องใช้ความรู้ใดในการทำงานบ้าง

\section{\ifenglish%
Impacts of this project on society, health, safety, legal, and cultural issues
\else%
ผลกระทบด้านสังคม สุขภาพ ความปลอดภัย กฎหมาย และวัฒนธรรม
\fi}

แนวทางและโยชน์ในการประยุกต์ใช้งานโครงงานกับงานในด้านอื่นๆ รวมถึงผลกระทบในด้านสังคมและสิ่งแวดล้อมจากการใช้ความรู้ทางวิศวกรรมที่ได้
