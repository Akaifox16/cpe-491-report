\chapter{\ifenglish Introduction\else บทนำ\fi}

\section{\ifenglish Project rationale\else ที่มาของโครงงาน\fi}

หอพักนักศึกษามหาวิทยาลัยเชียงใหม่ เป็นหอพักที่นักศึกษาชั้นปีที่ 1 ส่วนมากให้ความสนใจ เนื่องจากมีราคาถูก และตั้งอยู่ในเขตมหาวิทยาลัยที่มีโครงสร้างพื้นฐานมากมาย
ระบบการจองหอพักในปัจจุบันนั้น ประกอบด้วยหลายขั้นตอน ได้แก่
\begin{enumerate}
\item \CI{ลงชื่อเข้าใช้งาน}{ต้อง register ไหม?}
\item เลือกห้องพักที่ต้องการพักอาศัย
\item เลือกห้องพักที่ต้องการพักอาศัย
\item ยืนยันการลงทะเบียน
\end{enumerate}
ในแต่ละขั้นตอน มีการส่ง \CI{request}{แปลว่า?} เพื่อเรียกหน้าเว็บไซต์แต่ละหน้า ทำให้ระบบการจองหอพักไม่สามารถรองรับผู้ใช้งานจำนวนมากได้ในช่วงเวลาหนึ่งๆ กล่าวคือ เมื่อมีจำนวนผู้ใช้งานเป็นจำนวนมากจะทำให้ระบบไม่ตอบสนองแล้วไม่มี \CI{package}{?} ส่งกลับมาให้ผู้ใช้งาน
นอกจากนี้ เว็บไซต์ยังต้องประมวลผลการลงทะเบียนตลอดเวลา
หากห้องที่นักศึกษาต้องการเลือกนั้นไม่ว่างในหอหนึ่งๆ หรือหอที่ต้องการเลือกนั้นเต็มแล้ว นักศึกษาจำเป็นต้องย้อนกลับไปยังหน้าก่อนหน้า ซึ่งเป็นการส่ง request ใหม่ไปยังเซิฟเวอร์เช่นกัน
\CI{ส่งผลให้นักศึกษาต้องรีบลงทะเบียนให้ได้อยู่ในหอที่ต้องการ}{supporting evidence?} \CI{หลังจากนั้นจึงไปแลกห้องให้ได้อยู่กับเพื่อนที่ต้องการ}{evidence?} ซึ่งอาจจะทำให้\CI{เพื่อนไม่ได้อยู่หอเดียวกัน}{?} หรือห้องเดียวกัน ทำให้เพื่อนร่วมห้องที่ได้นั้นมาจากการสุ่ม ซึ่งอาจจะทำให้ไม่เข้ากัน
นำไปสู่การทะเลาะกัน หรืออยู่ด้วยกันแบบอึดอัดใจกัน
\CIreply{ต้นตอปํญหาคืออะไรกันแน่ ที่เขียนมาค่อนข้างจะ claim เยอะ แต่ยังไม่ค่อยเห็นเหตุผลสนับสนุน  ลองดูว่าปัญหาที่แท้จริงคืออะไร และระบบที่มีมันทำให้ปัญหาที่มีอยู่แล้วนั้นแย่ลง หรือเป็นตัวสร้างปัญหาตั้งแต่แรก จะได้แก้ปัญหาได้ตรงจุด}

ทางผู้จัดทำจึงได้คิดวิธีการแก้ปัญหาตัวระบบเก่าโดยการลดจำนวน request ที่ส่งมายัง server เพื่อแก้ปัญหา 
server รองรับ request จำนวนมากไม่ได้ ซึ่งระบบที่จะพัฒนาขึ้นใหม่นั้นจะไม่ประมวลผลระบบ ณ ขณะที่เปิดให้จอง
แต่จะรอให้ระบบปิดก่อนจึงจะประมวลผล และให้ผู้ใช้กรอกแบบสอบถามเพื่อนำมาใช้ในการเลือกเพื่อนร่วมห้องที่เหมาะสมที่สุด
\CIreply{ขายระบบมากเกินไป ควรขายเป้าหมายก่อน}

\section{\ifenglish Objectives\else วัตถุประสงค์ของโครงงาน\fi}
\begin{enumerate}
    \item พัฒนาเว็บแอพพลิเคชันตาม\CI{หลักการออกแบบ UX/UI}{?}
    \item ลดกระบวนการการคำนวณของ server\CIreply{อย่างไร?}
    \item จับคู่รูมเมท,หอพัก และ ห้องพักไม่ให้เกิด \CI{rouge couple}{?}
\end{enumerate}

\section{\ifenglish Project scope\else ขอบเขตของโครงงาน\fi}
\subsection{\ifenglish Hardware scope\else ขอบเขตด้านฮาร์ดแวร์\fi}
\begin{enumerate}
    \item เครื่องคอมพิวเตอร์ที่สามารถเชื่อมต่ออินเตอร์เน็ตได้
    \item \CI{เครื่องคอมพิวเตอร์ที่สามารถจัดเก็บข้อมูลการลงทะเบียนทั้งหมดได้}{can we use cloud?}
\end{enumerate}
\subsection{\ifenglish System scope\else ขอบเขตของระบบ\fi}
\begin{enumerate}
    \item ระบบจะรองรับเฉพาะการจองของหอพักนักศึกษามหาวิทยาลัยเชียงใหม่
    \item การจับคู่จะพิจารณาจากคุณสมบัติที่ระบบมีให้เท่านั้น
    \item ผู้ใช้จะไม่สามารถเลือกเฉพาะเจาะจงห้องพัก หรือ หอพักได้โดยตรง
\end{enumerate}
% ]

\section{\ifenglish Expected outcomes\else ประโยชน์ที่ได้รับ\fi}
\begin{enumerate}
    \item ผู้ใช้ไม่รู้สึก\CI{อึดอัด}{?}ขณะใช้งาน
    \item สังคมหอพักน่าอยู่ยิ่งขึ้น
    \item ภาพลักษณ์ที่\CI{ดี}{?}ของมหาวิทยาลัย
\end{enumerate}

\section{\ifenglish Technology and tools\else เทคโนโลยีและเครื่องมือที่ใช้\fi}

% \subsection{\ifenglish Hardware technology\else เทคโนโลยีด้านฮาร์ดแวร์\fi}
% \begin{enumerate}
%     \item 
%     \item 
%     \item 
% \end{enumerate}

\subsection{\ifenglish Software technology\else เทคโนโลยีด้านซอฟต์แวร์\fi}
\begin{enumerate}
    \item Typescript
    \item React.js
    \item Next.js
    \item Gin
    \item Go
    \item mySQL
    \item GraphQL
    \item Docker
\end{enumerate}
\CIreply{add reference and brief description to each tool; perhaps further separate into frontend/backend/etc.}

\section{\ifenglish Project plan\else แผนการดำเนินงาน\fi}
\begin{plan}{12}{2021}{2}{2023}
    \planitem{12}{2021}{1}{2022}{ศึกษาค้นคว้าอัลกอริทึม และ งานที่คล้ายคลึงกัน}
    \planitem{1}{2022}{1}{2022}{สอบถามข้อมูลจาก สำนักงานหอพัก}
    \planitem{1}{2022}{2}{2022}{รวบรวมข้อมูล สำหรับการทดสอบระบบ}
    \planitem{1}{2022}{2}{2022}{เลือกเครื่องมือในการพัฒนาระบบ}
    \planitem{2}{2022}{2}{2022}{ศึกษาเรียนรู้เกี่ยวกับเทคโนโลยี และ 
                                เครื่องมือที่ใช้พัฒนาระบบ}
    \planitem{3}{2022}{5}{2022}{ออกแบบเว็บแอพพลิเคชัน}
    \planitem{10}{2022}{12}{2022}{พัฒนาระบบฐานข้อมูล}
    \planitem{10}{2022}{1}{2023}{พัฒนาเว็บแอพพลิเคชัน}
    \planitem{1}{2023}{2}{2023}{ทดสอบระบบ}
    \planitem{2}{2023}{2}{2023}{เขียนรายงานสรุปผลของการพัฒนาระบบ}
\end{plan}

% \section{\ifenglish Roles and responsibilities\else บทบาทและความรับผิดชอบ\fi}
% อธิบายว่าในการทำงาน นศ. มีการกำหนดบทบาทและแบ่งหน้าที่งานอย่างไรในการทำงาน จำเป็นต้องใช้ความรู้ใดในการทำงานบ้าง

\section{\ifenglish%
Impacts of this project on society, health, safety, legal, and cultural issues
\else%
ผลกระทบด้านสังคม สุขภาพ ความปลอดภัย กฎหมาย และวัฒนธรรม
\fi}

แนวทางและโยชน์ในการประยุกต์ใช้งานโครงงานกับงานในด้านอื่นๆ รวมถึงผลกระทบในด้านสังคมและสิ่งแวดล้อมจากการใช้ความรู้ทางวิศวกรรมที่ได้
