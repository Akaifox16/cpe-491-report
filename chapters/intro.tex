\chapter{\ifenglish Introduction \else บทนำ\fi}
\section{\ifenglish Project rationale\else ที่มาของโครงงาน\fi}

% \MEreply{
%     \begin{itemize}
%         \item describe current system, user flow ans result
%         \item how system cause the issues user acceptance
%         \item -----------"--------------- traffic then load then low avaiability make even worse
%         \item how to solve this problem
%     \end{itemize}
% }
หอพักนักศึกษามหาวิทยาลัยเชียงใหม่สำหรับนักศึกษาชั้นปีที่ 1 ประกอบไปด้วยห้องพักที่มีผู้ร่วมอาศัย 2–4 คน
โดยระบบจองหอพักที่ให้บริการในปัจจุบัน จะให้นักศึกษาใหม่ที่ได้รับคัดเลือกในแต่ละรอบของระบบการคัดเลือกนักศึกษาแบบ TCAS เข้าไปเลือกหอพักตามช่วงเวลาที่กำหนด
การเปิดจองหอพักในแต่ละรอบ มีขั้นตอนการทำงานดังต่อไปนี้
\begin{enumerate}
    \item นักศึกษาลงชื่อเข้าใช้งานระบบ
    \item นักศึกษาเลือกหอพักที่ต้องการจอง
    \item นักศึกษาเลือกห้องพักที่ต้องการพักอาศัย
    \item กดยืนยันการจองห้องพัก
\end{enumerate}
แม้ขั้นตอนดังกล่าวจะเรียบง่ายและไม่ซับซ้อน แต่การเปิดระบบเป็นหลายรอบๆ ให้สอดคล้องกับรอบของ TCAS นั้น ทำให้ผู้ดูแลหอพักต้องแบ่งสำรองจำนวนห้องพักที่สามารถจองได้ในแต่ละรอบ
ส่งผลให้ในแต่ละรอบจะมีห้องว่างที่นักศึกษาที่สอบผ่านในรอบแรกๆ ไม่สามารถจองได้ แม้ว่าห้องพักดังกล่าวจะเป็นห้องที่ตรงความต้องการของนักศึกษา 
และยังปิดกั้นไม่ให้นักศึกษาที่ได้รับคัดเลือกในรอบ TCAS ที่ต่างกันได้มีโอกาสพักอาศัยด้วยกันอีกด้วย

นอกจากนักศึกษาอาจจะไม่ได้ชนิดของห้องพักที่ตนเองต้องการแล้ว ยังมีประเด็นในเรื่องของผู้ร่วมอาศัยในแต่ละห้อง
เนื่องจากในขั้นตอนการเลือกห้องพักนั้น นักศึกษาจะเห็นเพียงรายชื่อของนักศึกษาคนอื่นๆ ที่จองห้องพักเดียวกัน หากนักศึกษาไม่รู้จักใครเลยในรายชื่อดังกล่าว จะทำให้การเลือกห้องพักเป็นการตัดสินใจโดยขาดข้อมูลที่จำเป็นเกี่ยวกับผู้ร่วมอาศัยในแต่ละห้อง เปรียบเสมือนเป็นการสุ่มเพื่อนร่วมห้องไปโดยปริยาย

การสุ่มเพื่อนร่วมห้องโดยปริยาย อาจก่อปัญหาความขัดแย้งภายในห้อง อันเนื่องจากวิถีชีวิต กิจวัตรประจำวัน ตารางเวลา หรือแม้แต่บุคลิกภาพและนิสัยส่วนตัวที่ไม่เข้ากันในการเป็นผู้ร่วมอาศัย
แม้ว่านักศึกษาจะสามารถร้องขอผู้ดูแลหอพักเพื่อเปลี่ยนห้องหรือเพื่อนร่วมห้องในภายหลังได้ แต่ก็แลกมาด้วยภาระของผู้ดูแลหอพักที่เพิ่มขึ้น เพราะผู้ดูแลหอพักต้องดำเนินการแก้ไขข้อมูลหลังบ้านเป็นรายๆ ไป ซึ่งเป็นการจัดการนอกระบบ

นอกเหนือจากปัญหาการตัดสินใจโดยขาดข้อมูลแล้ว ระบบจองหอพักนักศึกษามหาวิทยาลัยเชียงใหม่ในปัจจุบันยังใช้วิธีการแบบมาก่อนได้ก่อน กล่าวคือ นักศึกษาใหม่ที่เข้ามาเลือกห้องช้ากว่าคนอื่น อาจประสบปัญหาขาดแคลนห้องพักที่ตรงตามความต้องการของตนเอง และถูกบีบบังคับให้เลือกห้องพักที่ยังคงเหลืออยู่เท่านั้น
แม้ว่าสถานการณ์ดังกล่าวอาจจะฟังดูยุติธรรมจากมุมมองด้านความรับผิดชอบของตัวนักศึกษาเองที่เข้ามาดำเนินการช้ากว่าคนอื่นๆ แต่ก็ทำให้ผู้ที่เข้ามาจองห้องพักก่อนนั้นได้รับผลกระทบไปด้วย โดยอาจจะได้ร่วมพักอาศัยกับนักศึกษาคนดังกล่าว

อย่างไรก็ตาม ด้วยธรรมชาติของกระบวนการจองห้องพักนั้น จะเห็นว่า ผู้ที่เข้ามาจับจองห้องพักก่อน มีโอกาสที่จะได้ห้องที่ตนเองต้องการมากกว่า
จึงทำให้เกิดการแข่งขันแย่งชิงกันภายในระบบ
และอาจทำให้นักศึกษาที่ต้องการพักอาศัยด้วยกันสูญเสียโอกาสในการได้อยู่ห้องเดียวกัน เป็นการเพิ่มความยากลำบากในการเลือกห้องพัก และก่อให้เกิดความตึงเครียดจากกระบวนการนี้ได้

ในท้ายที่สุด การแข่งขันในการเข้าระบบ อาจทำให้ในขณะหนึ่งๆ มีผู้ใช้ระบบมากเกินความสามารถของเซิร์ฟเวอร์ ก่อให้เกิดปัญหาทางเทคนิค เช่น 
มีการเชื่อมต่อฐานข้อมูลมากเกินไป และเว็บแอพพลิเคชันไม่สามารถให้บริการได้ เป็นต้น ทำให้นักศึกษาใหม่ไม่สามารถเข้าใช้งานระบบได้จนสิ้นสุดกระบวนการ 
เกิดความรำคาญต่อตัวระบบ และอาจผิดหวังในด้านการจัดการของมหาวิทยาลัยในองค์รวมได้

เพื่อแก้ไขปัญหาดังกล่าว โครงงานนี้จึงได้พัฒนาแอพพลิเคชันจองหอพัก 
ที่เพิ่มเติมส่วนการอนุญาตให้นักศึกษาระบุคุณลักษณะของหอพัก ห้องพัก 
และเพื่อนร่วมห้องที่ต้องการ
โดยเปิดให้ใช้บริการระบบนานขึ้น และลำดับการเข้าใช้งานระบบก่อน/หลัง จะไม่ส่งผลอย่างมีนัยสำคัญต่อระดับความพึงพอใจในห้องพักที่ผู้ใช้จะได้รับจัดสรร
หลังจากปิดระบบ จะมีการจับคู่เพื่อนร่วมห้องให้ผู้ใช้ทุกคนแบบอัตโนมัติ 
โดยพิจารณาจากความคล้ายคลึงกันของความต้องการของแต่ละฝ่าย

ด้วยการที่แอพพลิเคชันนี้มีข้อมูลเกี่ยวกับเพื่อนร่วมห้องที่ผู้ใช้ต้องการมากกว่าระบบปัจจุบัน และอนุญาตให้ผู้ใช้ระบุความต้องการของตนเอง 
จะทำให้การจับคู่เพื่อนร่วมห้องที่เหมาะสมเป็นปัญหาที่ง่ายขึ้น และสามารถบรรเทาปัญหาความขัดแย้งได้ 
นอกจากนี้ การขยายช่วงเวลาที่เปิดให้ใช้งานระบบ ผู้ใช้จะมีเวลาในการตัดสินใจมากขึ้น 
จึงสามารถลดระดับการแข่งขัน เพิ่มโอกาสในการหาเพื่อนร่วมห้องที่เหมาะสม และยังส่งผลให้ตัวระบบเองไม่ต้องแบกรับภาระในการรองรับจำนวนนักศึกษาที่เข้าใช้งานระบบพร้อมๆ กันได้ 
โดยสรุป วิธีการดังกล่าวจะสามารถยกระดับประสบการณ์การจองหอพัก 
และเลือกผู้ร่วมอาศัยของนักศึกษามหาวิทยาลัยเชียงใหม่ให้ดีขึ้นได้

\section{\ifenglish Objectives\else วัตถุประสงค์ของโครงงาน\fi}
\begin{enumerate}
    \item เพื่อพัฒนาเว็บแอพพลิเคชันสำหรับการจองหอพักที่อนุญาตให้ผู้ใช้ระบุ...
    \item ลดภาระงานของระบบฐานข้อมูล ในช่วงเปิดให้ใช้งาน
    \item เพิ่มความสามารถของระบบในการจับคู่เพื่อนร่วมห้องที่เหมาะสม
    \item เพื่อลดความแออัดในการในการเข้าใช้งานระบบ และฐานข้อมูล
\end{enumerate}

\section{\ifenglish Project scope\else ขอบเขตของโครงงาน\fi}
\subsection{\ifenglish Hardware scope\else ขอบเขตด้านฮาร์ดแวร์\fi}
\begin{enumerate}
    \item เครื่องเซิร์ฟเวอร์ที่ติดตั้ง Docker engine
\end{enumerate}
\subsection{\ifenglish System scope\else ขอบเขตของระบบ\fi}
\begin{enumerate}
    \item ระบบจะรองรับเฉพาะการจองของหอพักนักศึกษามหาวิทยาลัยเชียงใหม่
    \item การจับคู่ผู้ร่วมอาศัยจะพิจารณาจากคุณสมบัติระดับของความสะอาด
          ระดับความดังของเสียงกรน และช่วงเวลาที่งดใช้เสียงในห้องเท่านั้น
    \item การพิจารณาห้องของนักศึกษา จะพิจารณาชั้นของห้อง ระยะทางไปห้องน้ำ และระยะทางห้องพักกับบันไดเท่านั้น
    \item การพิจารณาหอพักของนักศึกษา จะพิจารณาค่าห้องพัก และจำนวนเพื่อนร่วมห้องเท่านั้น
    \item ผู้ใช้จะไม่สามารถเลือกเฉพาะเจาะจงห้องพัก หรือ หอพักได้โดยตรง
    \item สามารถรองรับจำนวนนักศึกษาพร้อมๆ กันสูงสุด 10,000 คน
\end{enumerate}
% ]

\section{\ifenglish Expected outcomes\else ประโยชน์ที่ได้รับ\fi}
\begin{enumerate}
    % \item ระบบถูกนำไปใช้งานได้จริง
    \item ผู้ใช้สามารถระบุความต้องการเพื่อให้ระบบจับคู่ได้เหมาะสมกว่าที่เป็นอยู่ในปัจจุบัน
    \item ผู้ใช้มีความพึงพอใจในผลลัพธ์การจับคู่
    \item ผู้ใช้งานขอเปลี่ยนห้องด้วยวิธีนอกระบบลดลง
    \item ผู้ดูแลไม่ต้องกลับมาจัดการหรือแก้ไขข้อมูลในระบบบ่อยๆ
\end{enumerate}

\section{\ifenglish Technology and tools\else เทคโนโลยีและเครื่องมือที่ใช้\fi}

\subsection{\ifenglish Hardware technology\else เทคโนโลยีด้านฮาร์ดแวร์\fi}
\begin{enumerate}
    \item Docker~\cite{dke} -- เป็น container runtime engine ที่ช่วยสร้างและรัน containers จาก Docker images
\end{enumerate}

\subsection{\ifenglish Software technology\else เทคโนโลยีด้านซอฟต์แวร์\fi}
\begin{enumerate}
    \item TypeScript~\cite{typescript} -- ภาษาคอมพิวเตอร์ที่พัฒนาต่อยอดมาจากภาษา JavaScript
          ที่เน้นให้สามารถหาจุดที่จะทำให้เกิดข้อผิดพลาดได้ก่อนที่จะทำการรันแอพพลิเคชัน
    \item React~\cite{react-wiki} -- เป็น library ที่ใช้ในการพัฒนา front-end ของเว็บแอพพลิเคชัน
    \item Next.js~\cite{next-wiki} -- เป็น React Framework ที่ช่วยให้สามารถพัฒนาเว็บแอพพลิเคชันได้ง่ายยิ่งขึ้น
    % \item Go~\cite{golang} -- เป็นภาษาคอมพิวเตอร์ที่เข้าใจได้ง่าย และความสามารถที่เด่นในด้านการทำ concurrency
    \item PostgreSQL -- เป็นโปรแกรมจัดการฐานข้อมูล(DBMS) แบบฐานข้อมูลเชิงวัตถุสัมพันธ์ที่สามารถ จัดการข้อมูลด้วยภาษา SQL
    \item RPC~\cite{RPC-wiki} -- เป็นเครื่องมือในการตั้งข้อตกลงในการจัดส่งและสื่อสารแบบทางไกล
\end{enumerate}

% \section{\ifenglish Project plan\else แผนการดำเนินงาน\fi}
% \begin{plan}{12}{2021}{2}{2023}
%     \planitem{12}{2021}{1}{2022}{ศึกษาค้นคว้างานที่คล้ายคลึงกัน}
%     % \planitem{12}{2021}{1}{2022}{ศึกษาค้นคว้าอัลกอริทึม}
%     \planitem{1}{2022}{1}{2022}{สอบถามข้อมูลจากผู้ดูแล}
%     \planitem{1}{2022}{2}{2022}{รวบรวมข้อมูล สำหรับการทดสอบ}
%     \planitem{1}{2022}{2}{2022}{ศึกษาและเลือกเครื่องมือในการพัฒนา}
%     \planitem{3}{2022}{5}{2022}{ออกแบบแอพพลิเคชัน}
%     \planitem{10}{2022}{12}{2022}{พัฒนาฐานข้อมูล}
%     \planitem{10}{2022}{1}{2023}{พัฒนาแอพพลิเคชัน}
%     \planitem{1}{2023}{2}{2023}{ทดสอบระบบ}
%     \planitem{2}{2023}{2}{2023}{เขียนรายงานสรุปผลการพัฒนา}
% \end{plan}

% \section{\ifenglish Roles and responsibilities\else บทบาทและความรับผิดชอบ\fi}
% อธิบายว่าในการทำงาน นศ. มีการกำหนดบทบาทและแบ่งหน้าที่งานอย่างไรในการทำงาน จำเป็นต้องใช้ความรู้ใดในการทำงานบ้าง

\section{\ifenglish%
      Impacts of this project on society, health, safety, legal, and cultural issues
  \else%
      ผลกระทบด้านสังคม สุขภาพ ความปลอดภัย กฎหมาย และวัฒนธรรม
  \fi}
% แนวทางและโยชน์ในการประยุกต์ใช้งานโครงงานกับงานในด้านอื่นๆ 
% รวมถึงผลกระทบในด้านสังคมและสิ่งแวดล้อมจากการใช้ความรู้ทางวิศวกรรมที่ได้

โครงงานนี้มีส่วนช่วยในการลดเหตุขัดแย้งของผู้พักอาศัยตลอดช่วงเวลาที่พัก
เนื่องจากได้พักอาศัยกับเพื่อนร่วมห้องที่ต้องการ ยิ่งไปกว่านั้นแอพพลิเคชันยังสามารถช่วยส่งเสริมด้านสังคม 
และวัฒนธรรม ด้วยเหตุที่ผู้ใช้นั้นจะได้ผู้ร่วมอาศัยมาจากการจับคู่คนที่เหมาะสม ที่อาจจะไม่เคยรู้จักกันมาก่อน ทำให้มีโอกาสได้เจอเพื่อนใหม่ที่มีความชอบคล้ายๆกัน 
ได้สร้างเครือข่ายที่จะคอยส่งเสริมกันในอนาคต นอกจากนี้แอพพลิเคชันยังช่วยให้ผู้ดูแลไม่ต้องแบ่งจำนวนห้อง 
หรือรอบประมวลผลหลายๆ รอบ เพื่อรองรับกับระบบ TCAS ในการรับนักศึกษาอีกต่อไป
ผู้ดูแลจะไม่ต้องวิตกกังวลว่าระบบจะล่มทุกครั้งที่เปิดให้ประมวลผลหรือไม่ จึงทำให้ช่วยส่งเสริมสุขภาพร่างกาย 
และสุขภาพจิตของผู้ดูแลให้ดียิ่งขึ้น


% ในการทำโครงงานนี้ คาดว่านักศึกษาจะมีคุณภาพชีวิตที่ดียิ่งขึ้น ได้รู้จักเพื่อนใหม่ อยู่ร่วมกันอย่างมีความสุข
% นอกจากนี้อาจจะทำให้เกิดกิจกรรมใหม่ๆ ขึ้นในหอพัก เนื่องจากคนที่มีความสนใจที่คล้ายๆ กันได้มีโอกาสมาเจอกันมากยิ่งขึ้น
% ยิ่งไปกว่านั้น ทางมหาวิทยาลัยก็จะได้รับชื่อเสียงเพิ่มมากขึ้น เนื่องจากมีระบบการจัดการที่ช่วยให้นักศึกษามีความพึงพอใจในการพักอาศัยภายในหอพักของมหาวิทยาลัย
