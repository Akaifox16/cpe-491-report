\chapter{\ifenglish Introduction \else บทนำ\fi}
\section{\ifenglish Project rationale\else ที่มาของโครงงาน\fi}

\MEreply{use โครงการ not  โครงงาน}
\MEreply{
    \begin{itemize}
        \item describe current system, user flow ans result
        \item how system cause the issues user acceptance
        \item -----------"--------------- traffic then load then low avaiability make even worse
        \item how to solve this problem
    \end{itemize}
}
\ME{
    ในการจองหอพักของมหาวิทยาลัยเชียงใหม่นั้น มีขั้นตอนการทำงานดังต่อไปนี้
    \begin{enumerate}
        \item นักศึกษาลงชื่อเข้าใช้งานระบบ
        \item นักศึกษาเลือกหอพักที่ต้องการจอง
        \item นักศึกษาเลือกห้องพักที่ต้องการพักอาศัย
        \item กดยืนยันการจองห้องพัก
    \end{enumerate}
    ซึ่งเป็นระบบที่เรียบง่าย และไม่มีความซับซ้อน แต่ทว่าเพื่อให้สอดคล้องกับระบบการรับนักศึกษา TCAS 
    ที่แบ่งรอบการรับนักศึกษาเป็นหลายๆ รอบ ทำให้ผู้ดูแลหอพักต้องแบ่งสำรองจำนวนห้องพักที่สามารถจองได้ในแต่ละรอบ
    ทำให้ในแต่ละรอบนั้นจะมีห้องว่างที่ไม่สามารถเปิดให้ นักศึกษาที่สอบผ่านเป็นรอบแรกๆ ได้จองห้องเหล่านั้น 
    จึงกลายเป็นปัญหาขึ้นในระบบการจองหอพัก
}
{first item of guideline}

ระบบจองหอพักนักศึกษามหาวิทยาลัยเชียงใหม่ประสบปัญหาหลายประการ ที่อาจทำให้นักศึกษายากที่จะจองหอพัก 
และหาห้องพักที่เหมาะสม ปัญหาหนึ่งคือความไม่เพียงพอของข้อมูลเกี่ยวกับผู้เข้าพักร่วม 
เนื่องจากในขั้นตอนการจองห้องนั้นนักศึกษาจะมองเห็นเพียงชื่อของนักศึกษาที่จองห้องพักเดียวกัน
นักศึกษาที่ไม่มีเพื่อนที่จะอยู่ร่วมกันต้องหาผู้เข้าพักร่วมโดยการสุ่ม ซึ่งอาจทำให้เกิดปัญหาความขัดแย้ง 
และยากที่จะประสบความสัมพันธ์ที่ดีต่อกัน ปัญหาความขัดแย้งนี้สามารถเกิดขึ้นได้จากวิถีชีวิต กิจวัตรประจำวัน 
ตารางเวลา หรือแม้แต่บุคคลิกภาพ และนิสัยส่วนตัว ที่ไม่เข้ากันในการเป็นเพื่อนร่วมห้อง
อย่างไรก็ตามเมื่อนักศึกษาทราบผลการจับห้องแล้ว แต่ไม่พึงพอใจสามารถแจ้งไปยังผู้ดูแลหอพัก 
เพื่อขอแลกห้อง หรือเพื่อนร่วมห้องได้ ซึ่งเป็นการบรรเทาปัญหาข้างต้น โดยแลกกับภาระที่เพิ่มขึ้นของผู้ดูแลหอพัก
เพราะผู้ดูแลหอพักนั้นต้องทำการเข้าไปแก้ไขข้อมูลภายในด้วยตนเอง ในการย้ายห้องพักให้กับนักศึกษาทุกคนที่ยื่นคำร้อง

ระบบนั้นประสบปัญหาเรื้อรังจาก การแข่งขันที่สูงของระบบการจอง นักศึกษาถูกบีบบังคับให้รีบตัดสินใจในการจองห้องพัก
ซึ่งนำพาไปสู่การตัดสินใจเลือก และมีความเป็นไปได้ที่จะนำไปสู่ปัญหาการทะเลาะวิวาทมากขึ้น ด้วยที่ระบบนั้นมีการแข่งขันที่สูงเป็นธรรมชาติอยู่แล้วนั้น
ทำให้โอกาสที่นักศึกษาที่รู้จักกันมาก่อน จะสามารถลงทะเบียนเลือกต้องเดียวกัน ส่งผลให้ยิ่งทวีความตึงเครียด
และความยากลำบากในการเลือกที่พักอาศัย

นอกจากนี้ความต้องการที่สูงก่อให้เกิดปัญหาทำให้มีผู้ใช้มากเกินความสามารถของเซิร์ฟเวอร์ ก่อให้เกิดปัญหาทางเทคนิคเช่น 
เว็บแอพพลิเคชันไม่สามารถให้บริการได้ ทำให้ผู้ใช้งาน หรือนักศึกษาไม่สามารถเข้าใช้งานได้จนสิ้นสุดการใช้งาน 
ก่อให้เกิดความรำคาญ และความยากลำบากของผู้ใช้งาน

เพื่อแก้ไขปัญหาดังกล่าว โครงงานนี้จึงได้พัฒนาแอพพลิเคชันจองหอพัก 
ที่เพิ่มเติมส่วนการอนุญาตให้นักศึกษาระบุคุณลักษณะของห้อง 
และเพื่อนร่วมห้องที่ต้องการ เพราะปัจจุบันนักศึกษาเห็นเพียงชื่อของเพื่อนร่วมห้อง
โดยเปิดให้ใช้บริการระบบนานขึ้น หลังจากปิดระบบ จะมีการจับคู่ให้ผู้ใช้ทุกคนแบบอัตโนมัติ 
โดยอ้างอิงจากความคล้ายคลึงกันของความต้องการของแต่ละฝ่าย
ด้วยการที่แอพพลิเคชันให้ข้อมูลเกี่ยวกับรูมเมทที่จะได้มากขึ้น และอนุญาตให้ผู้ใช้ระบุความต้องการของตนเอง 
จะทำให้ง่ายต่อระบบที่จะจับคู่เพื่อนร่วมห้องที่เหมาะสม และสามารถหลีกเลี่ยงปัญหาความขัดแย้งได้ 
อีกด้วยว่าการขยายช่วงเวลาที่เปิดให้ใช้งานระบบยาวนานขึ้น ผู้ใช้จะมีเวลามากขึ้นในการตัดสินใจ 
จึงสามารถลดระดับการแข่งขัน และเพิ่มโอกาสในการหาคู่ที่เหมาะสม ในการลดระดับการแข่งขันลงได้นั้น 
ยังสามารถช่วยลดภาระงานในการรองรับจำนวนนักศึกษาที่เข้าใช้งานระบบพร้อมๆ กันได้ 
ในท้ายนี้วิธีการนี้นั้นจะสามารถพัฒนาประสบการณ์การจองหอพัก 
และเลือกผู้ร่วมอาศัยของนักศึกษามหาวิทยาลัยเชียงใหม่ให้ดีขึ้นได้

\section{\ifenglish Objectives\else วัตถุประสงค์ของโครงงาน\fi}
\begin{enumerate}
    \item เพื่อพัฒนาเว็บแอพพลิเคชันสำหรับการจองหอพักที่อนุญาตให้ผู้ใช้ระบุ...ภาระงานของระบบฐานข้อมูล ในช่วงเปิดให้ใช้งาน
    \item เพิ่มความสามารถของระบบในการจับคู่เพื่อนร่วมห้องที่เหมาะสม
    \item เพื่อลดความแออัดในการในการเข้าใช้งานระบบ และฐานข้อมูล
\end{enumerate}

\section{\ifenglish Project scope\else ขอบเขตของโครงงาน\fi}
\subsection{\ifenglish Hardware scope\else ขอบเขตด้านฮาร์ดแวร์\fi}
\begin{enumerate}
    \item เครื่องเซิร์ฟเวอร์ที่ติดตั้ง Docker engine
\end{enumerate}
\subsection{\ifenglish System scope\else ขอบเขตของระบบ\fi}
\begin{enumerate}
    \item ระบบจะรองรับเฉพาะการจองของหอพักนักศึกษามหาวิทยาลัยเชียงใหม่
    \item \ME{การจับคู่จะพิจารณาจากคุณสมบัติที่ระบบมีให้เท่านั้น}{คุณสมบัติใดบ้าง}
    \item ผู้ใช้จะไม่สามารถเลือกเฉพาะเจาะจงห้องพัก หรือ หอพักได้โดยตรง
    \item \MEreply{concurrent user specification?}
\end{enumerate}
% ]

\section{\ifenglish Expected outcomes\else ประโยชน์ที่ได้รับ\fi}
\begin{enumerate}
    \item ระบบถูกนำไปใช้งานได้จริง
    \item ผู้ใช้สามารถระบุความต้องการเพื่อให้ระบบจับคู่ได้เหมาะสมกว่าที่เป็นอยู่ในปัจจุบัน
    \item ผู้ใช้มีความพึงพอใจในผลลัพธ์การจับคู่
    \item ผู้ใช้งานขอเปลี่ยนห้องด้วยวิธีนอกระบบลดลง
    \item \ME{ผู้ดูแลไม่ต้องกลับมาจัดการหรือแก้ไขข้อมูลในระบบบ่อยๆ}{ให้บทนำอธิบายให้เคลียร์}
\end{enumerate}

\section{\ifenglish Technology and tools\else เทคโนโลยีและเครื่องมือที่ใช้\fi}

\subsection{\ifenglish Hardware technology\else เทคโนโลยีด้านฮาร์ดแวร์\fi}
\begin{enumerate}
    \item Docker~\cite{dke} -- เป็น container runtime engine ที่ช่วยสร้างและรัน containers จาก Docker images
\end{enumerate}

\subsection{\ifenglish Software technology\else เทคโนโลยีด้านซอฟต์แวร์\fi}
\begin{enumerate}
    \item TypeScript~\cite{typescript} -- ภาษาคอมพิวเตอร์ที่พัฒนาต่อยอดมาจากภาษา JavaScript
          ที่เน้นให้สามารถหาจุดที่จะทำให้เกิดข้อผิดพลาดได้ก่อนที่จะทำการรันแอพพลิเคชัน
    \item React~\cite{react} -- เป็น library ที่ใช้ในการพัฒนา front-end ของเว็บแอพพลิเคชัน
    \item Next.js\cite{nextjs} -- เป็น React Framework ที่ช่วยให้สามารถพัฒนาเว็บแอพพลิเคชันได้ง่ายยิ่งขึ้น
    \item Go~\cite{golang} -- เป็นภาษาคอมพิวเตอร์ที่เข้าใจได้ง่าย และความสามารถที่เด่นในด้านการทำ concurrency
    \item PostgreSQL -- เป็นโปรแกรมจัดการฐานข้อมูล(DBMS) แบบฐานข้อมูลเชิงวัตถุสัมพันธ์ที่สามารถ จัดการข้อมูลด้วยภาษา SQL
    \item RPC -- เป็นเครื่องมือในการตั้งข้อตกลงในการจัดส่งและสื่อสารแบบทางไกล
\end{enumerate}
% \CIreply{add reference and brief description to each tool; perhaps further separate into frontend/backend/etc.}

% \section{\ifenglish Project plan\else แผนการดำเนินงาน\fi}
% \begin{plan}{12}{2021}{2}{2023}
%     \planitem{12}{2021}{1}{2022}{ศึกษาค้นคว้างานที่คล้ายคลึงกัน}
%     % \planitem{12}{2021}{1}{2022}{ศึกษาค้นคว้าอัลกอริทึม}
%     \planitem{1}{2022}{1}{2022}{สอบถามข้อมูลจากผู้ดูแล}
%     \planitem{1}{2022}{2}{2022}{รวบรวมข้อมูล สำหรับการทดสอบ}
%     \planitem{1}{2022}{2}{2022}{ศึกษาและเลือกเครื่องมือในการพัฒนา}
%     \planitem{3}{2022}{5}{2022}{ออกแบบแอพพลิเคชัน}
%     \planitem{10}{2022}{12}{2022}{พัฒนาฐานข้อมูล}
%     \planitem{10}{2022}{1}{2023}{พัฒนาแอพพลิเคชัน}
%     \planitem{1}{2023}{2}{2023}{ทดสอบระบบ}
%     \planitem{2}{2023}{2}{2023}{เขียนรายงานสรุปผลการพัฒนา}
% \end{plan}

% \section{\ifenglish Roles and responsibilities\else บทบาทและความรับผิดชอบ\fi}
% อธิบายว่าในการทำงาน นศ. มีการกำหนดบทบาทและแบ่งหน้าที่งานอย่างไรในการทำงาน จำเป็นต้องใช้ความรู้ใดในการทำงานบ้าง

\section{\ifenglish%
      Impacts of this project on society, health, safety, legal, and cultural issues
  \else%
      ผลกระทบด้านสังคม สุขภาพ ความปลอดภัย กฎหมาย และวัฒนธรรม
  \fi}
% แนวทางและโยชน์ในการประยุกต์ใช้งานโครงงานกับงานในด้านอื่นๆ 
% รวมถึงผลกระทบในด้านสังคมและสิ่งแวดล้อมจากการใช้ความรู้ทางวิศวกรรมที่ได้

โครงงานนี้มีส่วนช่วยในการลดเหตุขัดแย้งของผู้พักอาศัยตลอดช่วงเวลาที่พัก
เนื่องจากได้พักอาศัยกับเพื่อนร่วมห้องที่ต้องการ ยิ่งไปกว่านั้นแอพพลิเคชันยังสามารถช่วยส่งเสริมด้านสังคม 
และวัฒนธรรม ด้วยเหตุที่ผู้ใช้นั้นจะได้ผู้ร่วมอาศัยมาจากการจับคู่คนที่เหมาะสม ที่อาจจะไม่เคยรู้จักกันมาก่อน ทำให้มีโอกาสได้เจอเพื่อนใหม่ที่มีความชอบคล้ายๆกัน 
ได้สร้างเครือข่ายที่จะคอยส่งเสริมกันในอนาคต นอกจากนี้แอพพลิเคชันยังช่วยให้ผู้ดูแลไม่ต้องแบ่งจำนวนห้อง 
หรือรอบประมวลผลหลายๆ รอบ เพื่อรองรับกับระบบ TCAS ในการรับนักศึกษาอีกต่อไป
ผู้ดูแลจะไม่ต้องวิตกกังวลว่าระบบจะล่มทุกครั้งที่เปิดให้ประมวลผลหรือไม่ จึงทำให้ช่วยส่งเสริมสุขภาพร่างกาย 
และสุขภาพจิตของผู้ดูแลให้ดียิ่งขึ้น


% ในการทำโครงงานนี้ คาดว่านักศึกษาจะมีคุณภาพชีวิตที่ดียิ่งขึ้น ได้รู้จักเพื่อนใหม่ อยู่ร่วมกันอย่างมีความสุข
% นอกจากนี้อาจจะทำให้เกิดกิจกรรมใหม่ๆ ขึ้นในหอพัก เนื่องจากคนที่มีความสนใจที่คล้ายๆ กันได้มีโอกาสมาเจอกันมากยิ่งขึ้น
% ยิ่งไปกว่านั้น ทางมหาวิทยาลัยก็จะได้รับชื่อเสียงเพิ่มมากขึ้น เนื่องจากมีระบบการจัดการที่ช่วยให้นักศึกษามีความพึงพอใจในการพักอาศัยภายในหอพักของมหาวิทยาลัย
