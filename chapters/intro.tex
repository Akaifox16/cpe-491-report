\chapter{\ifenglish Introduction \else บทนำ\fi}
\section{\ifenglish Project rationale\else ที่มาของโครงงาน\fi}

หอพักนักศึกษามหาวิทยาลัยเชียงใหม่ เป็นหอพักที่นักศึกษาชั้นปีที่ 1 ส่วนมากให้ความสนใจ เนื่องจากมีราคาถูก และตั้งอยู่ในเขตมหาวิทยาลัยที่มีโครงสร้างพื้นฐานมากมาย
ระบบการจองหอพักในปัจจุบันนั้น ประกอบด้วยหลายขั้นตอน ได้แก่
\begin{enumerate}
    \item ลงทะเบียนสร้างบัญชีผู้ใช้
    \item ลงชื่อเข้าใช้งาน
    \item เลือกห้องพักที่ต้องการพักอาศัย
    \item เลือกห้องพักที่ต้องการพักอาศัย
    \item ยืนยันการลงทะเบียน
\end{enumerate}
ในแต่ละขั้นตอน มีการส่งคำร้อง(request)ไปยังเครื่องเซิฟเวอร์เพื่อเรียกหน้าเว็บไซต์แต่ละหน้า 
ทำให้ระบบการจองหอพักไม่สามารถรองรับผู้ใช้งานจำนวนมากได้ในช่วงเวลาหนึ่งๆ กล่าวคือเมื่อมีจำนวนผู้ใช้งานเป็นจำนวนมากจะทำให้ระบบไม่ตอบสนองต่อ 
request ของผู้ใช้นอกจากนี้ เว็บไซต์ยังต้องประมวลผลการลงทะเบียนตลอดเวลาหากห้องที่นักศึกษาต้องการเลือกนั้นไม่ว่างในหอหนึ่งๆ หรือหอที่ต้องการเลือกนั้นเต็มแล้ว 
นักศึกษาจำเป็นต้องย้อนกลับไปยังหน้าก่อนหน้า ซึ่งเป็นการส่ง request ใหม่ไปยังเซิฟเวอร์เช่นกัน 
อีกปัญหาหนึ่งที่มีจากระบบจองหอพักปัจจุบันนั่นคือการแข่งขันที่สูงเนื่องจากระบบจะจำกัดห้องให้จองได้แบ่งเป็นรอบๆทำให้แต่ละรอบที่เปิดนั้นมีผู้ใช้ที่จองไม่ทัน
และเนื่องจากการแข่งขันที่สูงส่งผลให้นักศึกษาต้องรีบลงทะเบียน ซึ่งอาจจะทำให้ไม่ได้อยู่หอหรือห้องเดียวกันกับเพื่อน ทำให้เพื่อนร่วมห้องที่ได้นั้นมาจากการสุ่ม ซึ่งอาจจะทำให้ไม่เข้ากัน
นำไปสู่การทะเลาะกัน หรืออยู่ด้วยกันแบบอึดอัดใจกัน
% \CIreply{ต้นตอปํญหาคืออะไรกันแน่ ที่เขียนมาค่อนข้างจะ claim เยอะ แต่ยังไม่ค่อยเห็นเหตุผลสนับสนุน  ลองดูว่าปัญหาที่แท้จริงคืออะไร และระบบที่มีมันทำให้ปัญหาที่มีอยู่แล้วนั้นแย่ลง หรือเป็นตัวสร้างปัญหาตั้งแต่แรก จะได้แก้ปัญหาได้ตรงจุด}

ทางผู้จัดทำจึงเลือกที่จะแก้ปัญหาการแข่งขันที่สูงของระบบจองหอพักที่มีในปัจจุบัน เพื่อให้ทุกคนมีสิทธิ์ที่จะได้จองอย่างเท่าเทียมกัน
โดยการลดความสามารถในการเฉพาะเจาะจงห้องหรือหอที่ต้องการได้ เป็นการบอกความต้องการคุณลักษณะของหอพักและห้องพักแทน
แล้วระบบใหม่จะพยายามเลือกให้ตรงตามความต้องการมากที่สุด อีกทั้งจะเพิ่มความสามารถในการจับคู่
เพื่อนร่วมห้องให้ตามความต้องการของผู้ใช้ ทำให้ผู้ใช้ที่อาจจะไม่มีเพื่อนที่อยู่ด้วยสามารถเลือกเพื่อนร่วมห้องที่เข้ากันได้และ
สามารถลดปัญหาการทะเลาะกันของการอยู่ร่วมกับเพื่อนร่วมห้องที่เข้ากันไม่ค่อยได้เนื่องจากเพื่อนร่วมห้องที่ได้จากการจับคู่มีความเข้ากันที่มากขึ้น
และจะแก้ปัญหาเซิร์ฟเวอร์ไม่ตอบสนองด้วยการลดจำนวนการประมวลผลการจองหอพักและการจับคู่ไปเป็นช่วงหลังจากปิดระบบลงทะเบียนแล้ว
% ทางผู้จัดทำจึงได้คิดวิธีการแก้ปัญหาตัวระบบเก่าโดยการลดจำนวน request ที่ส่งมายัง server เพื่อแก้ปัญหา 
% server รองรับ request จำนวนมากไม่ได้ ซึ่งระบบที่จะพัฒนาขึ้นใหม่นั้นจะไม่ประมวลผลระบบ ณ ขณะที่เปิดให้จอง
% แต่จะรอให้ระบบปิดก่อนจึงจะประมวลผล และให้ผู้ใช้กรอกแบบสอบถามเพื่อนำมาใช้ในการเลือกเพื่อนร่วมห้องที่เหมาะสมที่สุด
% \CIreply{ขายระบบมากเกินไป ควรขายเป้าหมายก่อน}

\section{\ifenglish Objectives\else วัตถุประสงค์ของโครงงาน\fi}
\begin{enumerate}
    \item พัฒนาเว็บแอพพลิเคชันตาม\CI{หลักการออกแบบ UX}{?}
    \item ลดกระบวนการการคำนวณของเซิร์ฟเวอร์ในวันที่ระบบเปิดให้ลงทะเบียน
    \item จับคู่รูมเมท, หอพัก และห้องพักไม่ให้เกิด \CI{rogue couple}{?}
\end{enumerate}

\section{\ifenglish Project scope\else ขอบเขตของโครงงาน\fi}
\subsection{\ifenglish Hardware scope\else ขอบเขตด้านฮาร์ดแวร์\fi}
\begin{enumerate}
    \item เครื่องคอมพิวเตอร์ที่สามารถเชื่อมต่ออินเตอร์เน็ตได้
\end{enumerate}
\subsection{\ifenglish System scope\else ขอบเขตของระบบ\fi}
\begin{enumerate}
    \item ระบบจะรองรับเฉพาะการจองของหอพักนักศึกษามหาวิทยาลัยเชียงใหม่
    \item การจับคู่จะพิจารณาจากคุณสมบัติที่ระบบมีให้เท่านั้น
    \item ผู้ใช้จะไม่สามารถเลือกเฉพาะเจาะจงห้องพัก หรือ หอพักได้โดยตรง
\end{enumerate}
% ]

\section{\ifenglish Expected outcomes\else ประโยชน์ที่ได้รับ\fi}
\begin{enumerate}
    \item ผู้ใช้ไม่เกิดความรู้สึกไม่สะดวกใจในการใช้งานระบบลงทะเบียน
    \item สังคมหอพักน่าอยู่ยิ่งขึ้น
    \item ชื่อเสียงด้านงานพัฒนาคุณภาพชีวิตนักศึกษาของมหาวิทยาลัยเพิ่มมากขึ้น
\end{enumerate}

\section{\ifenglish Technology and tools\else เทคโนโลยีและเครื่องมือที่ใช้\fi}

% \subsection{\ifenglish Hardware technology\else เทคโนโลยีด้านฮาร์ดแวร์\fi}
% \begin{enumerate}
%     \item 
%     \item 
%     \item 
% \end{enumerate}

\subsection{\ifenglish Software technology\else เทคโนโลยีด้านซอฟต์แวร์\fi}
\begin{enumerate}
    \item Typescript~\cite{typescript} - ภาษาคอมพิวเตอร์ที่พัฒนาต่อยอดมาจากภาษา Javascript 
        ที่เน้นให้สามารถหาจุดที่จะทำให้เกิดข้อผิดพลาดได้ก่อนที่จะทำการรันแอพพลิเคชัน
    \item React~\cite{react} - เป็น library ที่ใช้ในการพัฒนา Front-end ของเว็บแอพพลิเคชัน
    \item Next.js\cite{nextjs} - เป็น React Framework ที่ช่วยให้สามารถพัฒนาเว็บแอพพลิเคชันได้ง่ายยิ่งขึ้น
    \item Gin Gonic~\cite{gingonic} - เป็น Go Framework ที่ช่วยในการพัฒนา Back-end ของเว็บแอพพลิเคชัน
    \item Go~\cite{golang} - เป็นภาษาคอมพิวเตอร์ที่เข้าใจได้ง่าย และความสามารถที่เด่นในด้านการทำ concerrency
    \item MySQL~\cite{mysql} - เป็น relational database management system ที่มีความสามารถในการจัดการ transaction 
        และมีความนิยมในการใช้เป็นอย่างมาก
    \item GraphQL~\cite{graphql} -  เป็น query language Application Programming Interface (API) ที่มีลักษณะการใช้งานคล้ายคลึงกับการเขียน query เพื่อดึงข้อมูลจาก MySQL
    ทำให้ง่ายต่อการนำไปใช้ในการส่งข้อมูลและทำความเข้าใจ
    \item Docker~\cite{dke} เป็น container runtime engine ที่ช่วยสร้างและรัน container จาก Docker image
\end{enumerate}
% \CIreply{add reference and brief description to each tool; perhaps further separate into frontend/backend/etc.}

\section{\ifenglish Project plan\else แผนการดำเนินงาน\fi}
\begin{plan}{12}{2021}{2}{2023}
    \planitem{12}{2021}{1}{2022}{ศึกษาค้นคว้าอัลกอริทึม และ งานที่คล้ายคลึงกัน}
    \planitem{1}{2022}{1}{2022}{สอบถามข้อมูลจาก สำนักงานหอพัก}
    \planitem{1}{2022}{2}{2022}{รวบรวมข้อมูล สำหรับการทดสอบระบบ}
    \planitem{1}{2022}{2}{2022}{เลือกเครื่องมือในการพัฒนาระบบ}
    \planitem{2}{2022}{2}{2022}{ศึกษาเรียนรู้เกี่ยวกับเทคโนโลยี และ 
                                เครื่องมือที่ใช้พัฒนาระบบ}
    \planitem{3}{2022}{5}{2022}{ออกแบบเว็บแอพพลิเคชัน}
    \planitem{10}{2022}{12}{2022}{พัฒนาระบบฐานข้อมูล}
    \planitem{10}{2022}{1}{2023}{พัฒนาเว็บแอพพลิเคชัน}
    \planitem{1}{2023}{2}{2023}{ทดสอบระบบ}
    \planitem{2}{2023}{2}{2023}{เขียนรายงานสรุปผลของการพัฒนาระบบ}
\end{plan}

% \section{\ifenglish Roles and responsibilities\else บทบาทและความรับผิดชอบ\fi}
% อธิบายว่าในการทำงาน นศ. มีการกำหนดบทบาทและแบ่งหน้าที่งานอย่างไรในการทำงาน จำเป็นต้องใช้ความรู้ใดในการทำงานบ้าง

\section{\ifenglish%
Impacts of this project on society, health, safety, legal, and cultural issues
\else%
ผลกระทบด้านสังคม สุขภาพ ความปลอดภัย กฎหมาย และวัฒนธรรม
\fi}

% แนวทางและโยชน์ในการประยุกต์ใช้งานโครงงานกับงานในด้านอื่นๆ 
% รวมถึงผลกระทบในด้านสังคมและสิ่งแวดล้อมจากการใช้ความรู้ทางวิศวกรรมที่ได้
ในการทำโครงงานนี้ คาดว่านักศึกษาจะมีคุณภาพชีวิตที่ดียิ่งขึ้น ได้รู้จักเพื่อนใหม่ อยู่ร่วมกันอย่างมีความสุข
นอกจากนี้อาจจะทำให้เกิดกิจกรรมใหม่ๆ ขึ้นในหอพักเนื่องจากคนที่มีความสนใจที่คล้ายๆกันได้มีโอกาสมาเจอกันมากยิ่งขึ้น
ยิ่งไปกว่านั้นทางมหาลัยก็จะได้รับชื่อเสียงเพิ่มมากขึ้น เนื่องจากมีระบบการจัดการที่ช่วยให้
นักศึกษามีความพึงพอใจในการพักอาศัยในหอพักในนักศึกษา 