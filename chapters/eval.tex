\chapter{\ifproject%
\ifenglish Experimentation and Results\else การทดลองและผลลัพธ์\fi
\else%
\ifenglish System Evaluation\else การประเมินระบบ\fi
\fi}

ในบทนี้จะทดสอบเกี่ยวกับการทำงานในฟังก์ชันหลักๆ

\section{วิธีการตรวจสอบความถูกต้องของระบบ}
\subsection{นักศึกษา}
\begin{enumerate}
    \item ระบบยืนยันตัวตน: ทดสอบว่าผู้ใช้สามารถลงทะเบียน ยืนยันตัวตน และสามารถปิดกั้นการเข้าถึงส่วนของระบบที่ต้องผ่านการลงทะเบียนก่อนได้ถูกต้อง
    \item ระบบการจัดอันดับคุณลักษณะ: คุณลักษณะที่ถูกจัดอันดับถูกแปรเปลี่ยนเป็นตัวเลขที่นำไปใช้ในอัลกอริทึมมีการแปลงที่ถูกต้อง
    \item ระบบปรับจูนคุณลักษณะ: มีการปรับปรุงตัวเลขที่สะท้อนถึงคุณลักษณะที่ผู้ใช้ให้ความสมใจในโปรไฟล์ที่เลือกมา
            กล่าวคือ หากลักษณะใดๆ ที่พบเห็นว่าผู้ใช้ให้ความสนใจ ตัวคูณของคุณลักษณะนั้นควรเพิ่มขึ้น และเป็นเช่นเดียวกับในทางกลับกัน
    \item ระบบรายงานสรุปผล: สรุปผลที่ได้จากการจัดอันดับ และการปรับจูนได้ถูกต้องตามความเป็นจริง
    % \item ระบบคู่มือการใช้งาน: แสดงผลคู่มือการใช้งานที่มีลำดับขั้นตอนถูกต้องครบถ้วนสมบูรณ์
    % \item ระบบโปรไฟล์ผู้ใช้: มีการจัดเก็บถูกต้อง ตรงกับที่ผู้ใช้งานกรอกเข้าสู่ระบบ
\end{enumerate}

% \subsection{ผู้ดูแล}
% \begin{enumerate}
%     \item ระบบจัดการเวลาเปิด--ปิดวันลงทะเบียน: ระบบเปิด--ปิดวันลงทะเบียนได้ถูกต้องตามที่ตั้งค่าไว้
%     \item ระบบจัดการฐานข้อมูลและการตั้งค่าหอพักที่ใช้ในการลงทะเบียน: การตั้งค่าของหอและห้องพักที่เปิดให้ลงทะเบียนมีความถูกต้องตรงกับไฟล์ตั้งค่า
%     \item ระบบจัดการไฟล์ที่ใช้ในการจัดการระบบลงทะเบียน: เมื่อผู้ใช้ดาวน์โหลดไฟล์จะได้รับไฟล์ที่ต้องการได้ถูกต้อง
% \end{enumerate}

% \section{ประสิทธิภาพของอัลกอริทึม}
% \CIreply{load test?}
\section{วิธีการตรวจสอบความสามารถของระบบ}
% ในการทดสอบความสามารถนั้น จะทำการวัดประสิทธิภาพในการรับภาระงานของระบบ โดยทำการจำลองสถานการณ์ว่า หากมีผู้ใช้จำนวนมากๆ 
% ระบบจะสามารถรองรับได้ทั้งหมดกี่ requests ซึ่งระบบใหม่ควรรองรับจำนวน requests ได้ไม่น้อยกว่าระบบที่มีในปัจจุบัน
% \begin{enumerate}
%     % \item การทดสอบความรวดเร็วในการรับส่งข้อมูล: ทำการดึงข้อมูลจำนวนมากจากระบบ เพื่อวัดความเร็วในการรับส่งข้อมูล แล้วเปรียบเทียบกับระบบเก่า โดยจะต้องไม่ช้ากว่าที่มีอยู่
% \end{enumerate}
ในการทดสอบความสามารถนั้น จะวัดประสิทธิภาพโดยจำลองสถานการณ์ที่มีผู้เข้าใช้จำนวนมาก เพื่อทดสอบว่า ระบบจะสามารถรองรับ requests จากผู้ใช้ในจำนวนที่คาดหวังไว้ได้หรือไม่