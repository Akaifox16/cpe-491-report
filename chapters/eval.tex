\chapter{\ifproject%
\ifenglish Experimentation and Results\else การทดลองและผลลัพธ์\fi
\else%
\ifenglish System Evaluation\else การประเมินระบบ\fi
\fi}

ในบทนี้จะทดสอบเกี่ยวกับการทำงานในฟังก์ชันหลักๆ

\section{วิธีการตรวจสอบความถูกต้องของระบบ}
\subsection{End-user}
\begin{enumerate}
    \item ระบบยืนยันตัวตน: ทดสอบว่าผู้ใช้สามารถลงทะเบียน ยืนยันตัวตน และสามารถปิดกั้นการเข้าถึงส่วนของระบบที่ต้องผ่านการลงทะเบียนก่อนได้ถูกต้อง
    \item ระบบการจัดอันดับคุณลักษณะ: คุณลักษณะที่ถูกจัดอันดับถูกแปรเปลี่ยนเป็นตัวเลขที่นำไปใช้ในอัลกอริทึมมีการแปลงที่ถูกต้อง
    \item ระบบปรับจูนคุณลักษณะ: มีการปรับปรุงตัวเลขที่สะท้อนถึงคุณลักษณะที่ผู้ใช้ให้ความสมใจในโปรไฟล์ที่เลือกมา
            กล่าวคือหากลักษณะใดๆที่พบเห็นว่าผู้ใช้ให้ความสนใจ ตัวคูณของคุณลักษณะนั้นควรเพิ่มขึ้น และเป็นเช่นเดียวกับในทางกลับกัน
    \item ระบบรายงานสรุปผล: สรุปผลที่ได้จากการจัดอันดับ และการปรับจูนได้ถูกต้องตามความเป็นจริง
    \item ระบบคู่มือการใช้งาน: แสดงผลคู่มือการใช้งานที่มีลำดับขั้นตอนถูกต้องครบถ้วนสมบูรณ์
    \item ระบบโปรไฟล์ผู้ใช้: มีการจัดเก็บถูกต้อง ตรงกับที่ผู้ใช้งานกรอกเข้าสู่ระบบ
\end{enumerate}

\subsection{Administrator}
\begin{enumerate}
    \item ระบบจัดการเวลาเปิด--ปิดวันลงทะเบียน: ระบบเปิด--ปิดวันลงทะเบียนได้ถูกต้องตามที่ตั้งค่าไว้
    \item ระบบจัดการฐานข้อมูลและการตั้งค่าหอพักที่ใช้ในการลงทะเบียน: การตั้งค่าของหอและห้องพักที่เปิดให้ลงทะเบียนมีความถูกต้องตรงกับไฟล์ตั้งค่า
    \item ระบบจัดการไฟล์ที่ใช้ในการจัดการระบบลงทะเบียน: เมื่อผู้ใช้ดาวน์โหลดไฟล์จะได้รับไฟล์ที่ต้องการได้ถูกต้อง
\end{enumerate}

% \section{ประสิทธิภาพของอัลกอริทึม}
\CIreply{load test?}
\section{วิธีการตรวจสอบความสามารถของระบบ}
ในการทดสอบการความสามารถนั้นจะทำการวัดประสิทธิภาพในควา่มทนทานต่อการรับภาระงานหนักของระบบมากน้อยเพียงใด โดยมีการทดสอบดังนี้
\begin{enumerate}
    \item การทดสอบความสามารถในการรับ request จำนวนมาก : ทำการจำลองสถานการณ์ว่าหากมีผู้ใช้จำนวนมากๆ ระบบจะสามารถทนได้ทั้งหมดกี่ 
    request โดยระบบใหม่ควรรองรับได้ไม่น้อยกว่าระบบที่มีในปัจจุบัน
    \item การทดสอบความรวดเร็วในการส่งข้อมูล : ทำการดึงข้อมูลจำนวนมากจากระบบแล้ววัดความเร็วในการรับส่งข้อมูลแล้วเปรียบเทียบกับระบบเก่า โดยจะต้องไม่ช้ากว่าที่มีอย่
\end{enumerate}
