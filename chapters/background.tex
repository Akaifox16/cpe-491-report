\chapter{\ifenglish Background Knowledge and Theory\else ทฤษฎีที่เกี่ยวข้อง\fi}

การทำโครงงาน เริ่มต้นด้วยการศึกษาค้นคว้า ทฤษฎีที่เกี่ยวข้อง หรือ งานวิจัย/โครงงาน 
ที่เคยมีผู้นำเสนอไว้แล้ว ซึ่งเนื้อหาในบทนี้ก็จะเกี่ยวกับการอธิบายถึงสิ่งที่เกี่ยวข้องกับโครงงาน 
เพื่อให้ผู้อ่านเข้าใจเนื้อหาในบทถัดๆ ไปได้ง่ายขึ้น

\section{งานวิจัยที่คล้ายคลึงกัน}
\label{sec:rmp}
\subsection{Stable roommate matching problem}
Stable roommate matching problem เป็นปัญหาหนึ่งในกลุ่ม stable matching problem
โจทย์ของปัญหามีอยู่ว่า มีผู้เข้าร่วมอยู่ทั้งหมด $2n$ คนโดยทุกๆ คนจะทำการจัดอันดับความต้องการรูมเมทของตนเอง
เป็นจำนวน $2n-1$ อันดับ ซึ่งการจับคู่ผู้เข้าร่วมจะถูกนิยามเป็นคู่อันดับ $(m,n)$ ใดๆ โดยการจับคู่ $(m,n)$ ใดๆ จะเรียกว่าเป็น
stable matching ก็ต่อเมื่อ ไม่มีผู้เข้าร่วมคนอื่นคนใดนิยามว่า $n'$ กับ $m$ใดๆ ต้องการที่จะจับคู่กันมากกว่าคู่ที่ตนเองได้รับ ซึ่ง $(m, n')$
จะเรียกว่าเป็นคู่อุปสรรค(blocking pair หรือ rogue couple) ให้ไม่มี stable matching เกิดขึ้น

ซึ่งจากปัญหาข้างต้นนั้นจะเห็นได้ว่ามีลักษณะคล้ายคลึงกับโครงงานฉบับนี้และ ณ ขณะที่ศึกษา
โครงงานอยู่นั้นได้มีอัลกอริทึมที่สามารถแก้ปัญหานี้ได้แล้ว นั่นคือ Irving's algorithm
~\cite{irving1985efficient} ที่สามารถหาคำตอบได้ภายในเวลา $O(n^2)$

\subsection{NP-Problem}
ต่อไปนี้จะเป็นการอธิบายนิยามสั้นๆ ของกลุ่มปัญหา NP~\cite{np} เพื่อต่อยอดความรู้สู่ส่วนถัดไป
\begin{enumerate}
      \item NP-problem คือเซตของปัญหาที่สามารถหาคำตอบได้ด้วย Nondeterministic
            Turing Machine (NTM) ภายในเวลา polynomial time ตามขนาดของ input
      \item  NP-hard คือปัญหาที่สามารถนำวิธีแก้ปัญหานั้นๆ ไปช่วยแก้ปัญหา NP อื่นๆ ได้
      \item  NP-complete คือปัญหาที่สามารถพิสูจน์ได้ว่าเป็นทั้ง NP-problem และ NP-hard
\end{enumerate}
ซึ่งเมื่อใดก็ตามที่ปัญหาหนึ่งๆ ถูกสรุปว่าเป็น NP-complete ผู้แก้ปัญหามักจะไม่พยายามหาวิธีที่ดีที่สุด
แต่จะเปลี่ยนเป็นหาวิธีประมาณการแทน เนื่องจากปัญหา NP-complete 
ต้องใช้เวลาในการหาคำตอบที่ดีที่สุดเป็น super-polynomial~\cite{np}
ซึ่งเป็นเวลาที่นานไม่คุ้มกับเวลาที่เสียไป



\subsection{Stable roommates problem with triple rooms}
เนื่องจากทางหอพักของมหาวิทยาลัยเชียงใหม่มีห้องที่อยู่ร่วมกันมากกว่า 2 คน หากต้องการแก้ปัญหา stable roommate matching 
กับห้องที่มี 3 คน จะทำให้กลายเป็นstable roommate matching with triple rooms ซึ่งเป็นปัญหา 
NP-complete~\cite{iwama2007stable}ทำให้ใช้เวลานานในการหาคำตอบ ทางผู้จัดทำจึงเลือกที่จะไม่หาวิธีที่ดีที่สุด ในการแก้ปัญหานี้
แต่จะใช้ greedy algorithm แทนโดยจะขยายความการทำงานในบทที่ 3

\section{โครงสร้างเว็บแอพพลิเคชัน}
ส่วนต่อไปจากนี้จะเกี่ยวกับการจะเกี่ยวกับ การออกแบบโครงสร้างของเว็บแอพพลิเคชัน 
โดยจะเริ่มทำความรู้จักกับเว็บแอพพลิเคชันก่อนว่าคืออะไร
\subsection{เว็บแอพพลิเคชัน}
เว็บแอพพลิเคชันคือ แอพพลิเคชันที่ทำงานบนเครื่องเซิร์ฟเวอร์ โดยผู้ใช้งานสามารถเข้าใช้งานได้ผ่านเว็บเบราเซอร์ และระบบอินเทอร์เน็ต
ซึ่งเว็บแอพพลิเคชันถูกพัฒนาด้วยเว็บเทคโนโลยีต่างๆ เพื่อให้ทำงานได้หลากหลายรูปแบบ ตัวอย่างเว็บแอพพลิเคชันเช่น 
Facebook Youtube Spotify เป็นต้น

\subsection{แอพพลิเคชันแบบ Tree-tier}
โครงสร้าง 3 ขั้นหรือ Three-tier  นั้นถูกออกแบบให้แบ่งการทำงานของแอพพลิเคชัน เป็น 3 ส่วนใหญ่ๆ คือ
\begin{enumerate}
      \item ขั้นนำเสนอ(presentation teir) เป็นส่วนที่ผู้ใช้มีปฏิสัมพันธ์ด้วย
      \item ขั้นแอพพลิเคชัน(application tier) เป็นส่วนที่ใช้ในการประมวลผลข้อมูล
      \item ขั้นข้อมูล(data tier) เป็นส่วนที่ใช้ในการเก็บข้อมูลของแอพพลิเคชัน
\end{enumerate}
ซึ่งในส่วนของเว็บแอพพลิเคชันนั้นจะแบ่งแต่ละขั้น ออกเป็นส่วนต่างๆ คือ Front-end Back-end และ Database ตามลำดับ

ประโยชน์ที่ได้คือ ในแต่ละขั้นนั้นจะอิสระต่อกัน ทำให้ง่ายต่อการปรับเปลี่ยน 
หรือขยาย(scaling) โดยไม่กระทบกับส่วนอื่นๆ

ความนิยมในปัจจุบันนั้นมีแนวโน้มในการใช้ระบบ cloud computing ในการให้บริการแอพพลิเคชัน
ด้วยความช่วยเหลือจากเทคโนโลยี containers และ microservice ซึ่งผู้พัฒนาต้องการที่จะพัฒนาแอพพลิเคชัน 
ให้รองรับการเทคโนโลยี microservices
\subsection{สถาปัตยกรรม Microservice}
เป็นสถาปัตยกรรมการพัฒนาซอร์ฟแวร์ ที่แบ่งงาน หรือบริการต่างๆ ออกเป็นโปรแกรมย่อยๆ
เรียกว่า service โดยแต่ละ service จะมี database เป็นของตัวเอง และมีความอิสระต่อกัน 
ประโยชน์ที่ได้จากสถาปัตยกรรมคือ 
\begin{enumerate}
      \item สามารถทดแทนหรือบำรุง service ได้ง่ายเพราะทุก service อิสระต่อกันเหมือนกับโทรศัพท์ที่จอพัง 
      การทำงานส่วนอื่นก็ยังทำงานได้ปกติ หลังจากเปลี่ยนจอก็สามารถใช้งานต่อไป
      \item สามารถแยก scale แต่ละ service ได้เพราะ service ทุกตัวคือโปรแกรม 1 โปรแกรม เหมือนกับการใช้คอมพิวเตอร์
      ที่สามารถเปิดหน้าต่างเว็บเบราเซอร์ หรือโปรแกรมใดๆ ได้มากกว่า 1 หน้าต่างในเวลาเดียวกัน
\end{enumerate}

\section{Technology}
ในส่วนต่อไปนี้จะเป็นการอธิบายในเรื่องของเทคโนโลยีที่ใช้ในการพัฒนาเว็บแอพพลิเคชัน
\subsection{Front-end}
เทคโนโลยีในส่วนของ Front-end นั้นประกอบไปด้วย
เทคโนโลยีในการสร้างหน้าตาของเว็บ หรือ UI(user interface)
และเครื่องมือในการดึงข้อมูลจาก Back-end
\subsubsection{React}
เป็น open-source library ของภาษา javascript ในการพัฒนา UI พัฒนาขึ้นโดย Meta และชุมชนนักพัฒนา
จุดเด่นของ React~\cite{react-wiki} คืออนุญาตให้นักพัฒนาสามารถสร้างฟังก์ชันที่สามารถคืนค่ากลับมาเป็น html ได้ซึ่งฟังก์ชันนั้นจะเรียกว่า function component
\subsubsection{Next.js}
เป็น open-source React framework ในการพัฒนาเว็บแอพพลิเคชัน พัฒนาโดย Vercel
จุดเด่นของ Next.js~\cite{next-wiki} คือง่ายที่จะพัฒนาเว็บแอพพลิเคชันที่เรนเดอร์หน้าเว็บบนเซิร์ฟเวอร์(server-side rendering) และยังมีโปรแกรมในการสร้างหน้าเว็บนิ่ง(static-site generator)
รวมทั้ง Next.js ยังสามารถใช้ในการพัฒนาแอพพลิเคชันส่วน Back-end ได้อีกด้วย
\subsubsection{React-query}
เป็น React library ที่ช่วยในการพัฒนา client สำหรับการดึงข้อมูลจากส่วนของ Back-end ให้ง่ายยิ่งขึ้นด้วยฟังก์ชันที่พัฒนาไว้แล้วสำหรับการทำ cache 
หรือการดึงข้อมูลใหม่เมื่อเข้าเงื่อนใขบางอย่าง สามารถตรวจสอบได้ว่าการดึงข้อมูลมีการผิดพลาด หรือกำลังดึงข้อมูล หรือเสร็จสิ้นแล้ว ทำให้โค้ดของแอพพลิเคชันอ่านง่ายขึ้น 
และไม่ต้องเสียเวลาในการพัฒนาฟังก์ชันในการดึงข้อมูลต่างๆ เอง เนื่องจากแอพพลิเคชัน Three-tier นั้นส่วนของ Front-end และ Back-end ต้องติดต่อกันบ่อยครั้ง
\subsubsection{Typescript}
เป็นส่วนต่อขยาย(extension) ของ javascript โดยภาษา typescript~\cite{typescript} จะไม่อนุญาตให้นักพัฒนาใช้ตัวแปรเดิมในการเก็บตัวแปรต่างประเภทกันได้
อีกทั้งยังสามารถช่วยในการหาจุดที่อาจทำให้เกิดข้อผิดพลาดในช่วงพัฒนาได้ โดยที่ไม่ต้องทำการรันแอพพลิเคชันก่อน ซึ่งช่วยลดเวลาในการพัฒนาแอพพลิเคชันใหญ่ๆได้มาก
และที่สำคัญคือ typescript รองรับการทำงานร่วมกับ React Next.js และ React-query ด้วย
\subsection{Back-end}
เทคโนโลยีในส่วนของ Back-end นั้นประกอบไปด้วยส่วนของการประมวลผลข้อมูลต่างๆ ของระบบ และการให้บริการช่องทางการติดต่อกับ Front-end
\subsubsection{tRPC}
เป็น library ที่ช่วยในการพัฒนา API ที่ใช้ติดต่อกับ Front-end โดยหลักการทำงานจะเป็นการล้อวิธีการและชื่อมาจาก RPC ซึ่งจะอธิบายในส่วนต่อไป
โดย tRPC~\cite{trpcIntroducingTRPC} นั้นมีจุดเด่นคือสามารถระบุประเภทของข้อมูลว่ารับและส่งคืนเป็นตัวแปรประเภทใด และยังสามารถใช้ร่วมกับ React-query ได้ด้วย แต่ข้อเสียของ tRPC นั้นคือรองรับเพียง typescript
ในโครงงานนี้จึงให้ Next.js เป็น Back-end ร่วมด้วย
\subsubsection{RPC}
RPC~\cite{RPC-wiki} หรือ Remote Procedure Call หลักการการทำงานจะเป็นตามที่ระบุในชื่อ นั่นคือเป็นการเรียกใช้กระบวนการระยะไกล หรือก็คือการเรียกใช้งานฟังก์ชันจากเครื่องผู้ใช้ 
ให้เซิร์ฟเวอร์ประมวลผลแล้วส่งผลลัพธ์คืนกลับมาทางเครือข่ายอินเทอร์เน็ต ซึ่ง RPC เป็นวิธีการในการสื่อสารระหว่างเครื่องผู้ใช้ และเซิร์ฟเวอร์ที่เรียบง่ายเพราะไม่มีระเบียบวิธีซับซ้อน
จึงเหมาะที่จะนำมาใช้ในการพัฒนาแอพพลิเคชัน
% \subsubsection{Go}
% เป็นภาษาคอมพิวเตอร์ที่ถูกออกแบบโดย Google โดย Go ภาษาที่มีระบบตัวแปรแบบอพลวัต(statically type) และต้องคอมไพล์เป็นไฟล์ไบนารีก่อนที่จะรัน
% Go ถูกออกแบบมาให้เป็นภาษาที่สามารถเพิ่มประสิทธิผลในการพัฒนาโปรแกรม และข้อดีที่สุดของ Go คือความง่ายในการทำความเข้าใจการเขียนโปรแกรมแบบหลายเธรด(multithread) หรือ Concurrency
% และระบบจัดการ package และ version โปรแกรมที่สามารถดึง library ต่างๆ ได้จากผู้ให้บริการต่างๆ เช่น Github  หรือ Gitlab เป็นต้น
% \subsubsection{Concurrency}
% concurrency คือความสามารถที่ส่วนต่างๆ ของโปรแกรมสามารถทำงานอิสระจากกันได้ โดยที่ไม่กระทบซึ่งผลลัพธ์ ซึ่งจะสามารถเพิ่มประสิทธิภาพในการทำงานของโปรแกรมได้ 
% เพราะแต่ละส่วนย่อยๆ สามารถทำงานได้พร้อมๆ กัน ซึ่งหากสามารถย่อยส่วนต่างๆ ของอัลกอริทึมให้สามารถทำ concurrency ได้จะทำให้ประมวลผลการจับคู่ได้เร็วยิ่งขึ้น
\subsection{Deployment}
ส่วนต่อไปจากนี้จะเกี่ยวกับการ deploy เว็บแอพพลิเคชัน โดยเริ่มจากทำความรู้จักกับ 
containerization ว่าจริงๆ แล้วคืออะไร
\subsubsection{Containerization}
Containerization~\cite{ctnrh} คือการนำแอพพลิเคชันที่พัฒนา, library หรือสิ่งต่างๆ 
ที่จำเป็นในการทำงานของแอพพลิเคชันทั้งหมด บรรจุลงในกล่องกล่องเดียว โดยเรียกกล่องดังกล่าวว่า container 
เมื่อต้องการที่จะนำแอพพลิเคชันนั้นไปทดสอบในระบบต่างๆ เพียงเปิดกล่องดังกล่าวขึ้นมา
ก็จะมีอุปกรณ์ทุกอย่างที่พร้อมจะทำให้แอพพลิเคชันทำงานได้ปกติ การทำ containerization 
จึงมีความคล่องตัวในการ deploy แอพพลิเคชันในสภาพแวดล้อมต่างๆ ได้ดี 
\subsubsection{Docker}
Docker~\cite{dke} เป็นโปรแกรมที่ใช้ในการสร้างแม่แบบคอนเทนเนอร์ ซึ่ง Docker จะเรียกว่า Docker Image และใช้ในการรันตัวคอนเทนเนอร์
ซึ่งนอกจากจะสามารถรันคอนเทนเนอร์ได้แล้ว Docker ยังรองรับการทำเครือข่าย(Networking) ระหว่างคอนเทนเนอร์ด้วยกัน 
หรือจะทำการผูกระบบการจัดการไฟล์ของคอนเทนเนอร์เข้ากับเครื่องคอมพิวเตอร์ที่ใช้รัน เพื่อตรวจสอบไฟล์ต่างๆได้เช่นกัน
Docker ยังมีส่วนเสริมเรียกว่า Docker-compose ซึ่งสามารถช่วยในงานจัดการคอนเทนเนอร์ได้ดี เพราะสามารถระบุเปิดปิดคอนเทนเนอร์ที่ต้องทำงานร่วมกัน 
เชื่อมต่อเครือข่าย หรือ ผูกระบบไฟล์ต่างๆ ได้ในการระบุไฟล์รายละเอียดเรียกว่าไฟล์ docker-compose.yml เพียงไฟล์เดียว
\subsection{Infrastructure as Code}
Infrastructure as Code หรือ IaC เป็นซอร์ฟแวร์จัดการระบบโครงสร้างต่างๆ เช่น ฐานข้อมูล เน็ตเวิร์คพร็อกซ์ ฯลฯ 
ด้วยการระบุความต้องการสถานะ และการตั้งค่าต่างๆ เช่น หากต้องการตั้งค่าฐานข้อมูล จะสามารถระบุได้ว่าฐานข้อมูลนั้นมี ผู้ใช้กี่บัญชี 
มีสิทธิ์ในการจัดการอะไร มีฐานข้อมูลอะไรบ้าง เป็นต้น ตัวอย่างซอร์ฟแวร์ประเภทนี้ เช่น Terraform Ansible เป็นต้น
ซึ่งในโครงงานนี้จะใช้ Terraform ในการต้้งค่า PostgreSQL เพื่อลดความซับซ้อนและเพิ่มสะดวกในขั้นตอนการติดตั้งระบบ

\section{\ifenglish%
        \ifcpe CPE \else ISNE \fi knowledge used, applied, or integrated in this project
  \else%
        ความรู้ตามหลักสูตรซึ่งถูกนำมาใช้หรือบูรณาการในโครงงาน
  \fi
 }
\begin{enumerate}
      \item Database design: ทำให้มีความรู้พื้นฐานในการออกแบบและพัฒนาฐานข้อมูล
      \item Data Structures and Algorithms: นำความรู้ไปต่อยอดในการศึกษาและพัฒนาอัลกอริทึม
      \item Basic computer engineer laboratory: นำความรู้ไปต่อยอดการพัฒนาเว็บแอพพลิเคชัน
      \item Software Engineering: ช่วยในการวางแผน และระเบียบในการพัฒนา
\end{enumerate}


\section{\ifenglish%
        Extracurricular knowledge used, applied, or integrated in this project
  \else%
        ความรู้นอกหลักสูตรซึ่งถูกนำมาใช้หรือบูรณาการในโครงงาน
  \fi
 }
\begin{enumerate}
      \item Containerization: นำความรู้มาใช้ในการ deploy เว็บแอพพลิเคชัน
      \item Infrastructure as Code: ใช้ในการตั้งค่าฐานข้อมูลแบบอัตโนมัติ
      \item RPC: ใช้ในการเชื่อมต่อระบบฝั่ง Front-end และ Back-end เข้าด้วยกัน
\end{enumerate}
