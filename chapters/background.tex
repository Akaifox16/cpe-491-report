\chapter{\ifenglish Background Knowledge and Theory\else ทฤษฎีที่เกี่ยวข้อง\fi}

การทำโครงงาน เริ่มต้นด้วยการศึกษาค้นคว้า ทฤษฎีที่เกี่ยวข้อง หรือ งานวิจัย/โครงงาน 
ที่เคยมีผู้นำเสนอไว้แล้ว ซึ่งเนื้อหาในบทนี้ก็จะเกี่ยวกับการอธิบายถึงสิ่งที่เกี่ยวข้องกับโครงงาน 
เพื่อให้ผู้อ่านเข้าใจเนื้อหาในบทถัดๆ ไปได้ง่ายขึ้น

\section{UX design}
\subsection{User experience (UX)}
User experience (UX) คือความสะดวกและประสบการณ์การใช้งานของผู้ใช้ระบบ 
หรือแอพพลิเคชันนั้นๆ ว่าใช้งานง่ายมากน้อยเพียงใด ความสะดวก จำนวนครั้งที่ต้องมีปฏิสัมพันธ์กับระบบ
ความต่อเนื่องจากการกระทำหนึ่งๆ ไปยังการกระทำอีกอย่าง

\subsection{UX laws}
\label{subsec:uxlaws}
กฏในการออกแบบ UX~\cite{uxui} มี 11 ข้อดังนี้
\begin{enumerate}
  \item Aesthetic-Usability Effect -- อธิบายว่าผู้ใช้มักมองว่าความสวยงามน่าใช้คือความใช้ง่าย 
      ซึ่งแท้จริงแล้วอาจจะแอพพลิเคชันนั้นๆ อาจจะไม่ได้ใช้งานง่ายที่สุดก็ตาม ดังนั้นในการออกแบบพัฒนา ความสวยงามก็มีความจำเป็น
      เพราะไม่ว่าจะออกแบบให้ใช้ง่ายเพียงใด ก็จะถูกผู้ใช้งานตัดสินว่าใช้ยากจากหน้า user interface ที่ไม่สวยงาม
  \item Jakob's Law -- อธิบายว่าผู้ใช้มักคาดหวังให้เว็บที่พัฒนาใหม่นั้นทำงานเช่นเดียวกันกับเว็บที่เขาเคยมีประสบการณ์ใช้งาน
        การออกแบบระบบใหม่นี้จึงต้องออกแบบให้ส่วนประกอบต่างๆ ที่ผู้ใช้จะมีปฏิสัมพันธ์ มีหน้าตาคล้ายคลึงกับของเว็บไซต์อื่นๆ
        เพื่อให้ง่ายต่อการเรียนรู้ระบบใหม่
  \item Fitts's Law -- อธิบายว่าเวลาที่ผู้ใช้จะต้องใช้เพื่อปฏิสัมพันธ์กับฟังก์ชันการทำงาน จะขึ้นกับความใหญ่และระยะทางที่ต้องใช้เพื่อไปถึงเป้าหมาย
        ดังนั้น ส่วนประกอบที่สำคัญ จึงควรออกแบบให้มีขนาดที่มองเห็นได้ง่าย และในการจัดวางส่วนประกอบ หากส่วนใดต้องทำงานต่อเนื่องกัน ควรจัดวางให้อยู่ใกล้ๆ กัน
  \item Hick's Law -- อธิบายว่าเวลาที่ผู้ใช้จะต้องใช้เพื่อตัดสินใจเลือกบางอย่าง จะขึ้นอยู่กับจำนวนของตัวเลือกและความซับซ้อนของตัวเลือกนั้นๆ
  \item Miller's Law -- อธิบายว่าผู้ใช้สามารถจดจำทุกสิ่งทุกอย่างได้โดยเฉลี่ย $7\pm2$ อย่าง 
        เช่น เบอร์โทรหากเขียนว่า 09185XXXXX จะจำได้ยากกว่า 091-85X-XXXX เป็นต้น
  \item Tesler's Law -- เปรียบเทียบความซับซ้อนเป็นพลังงาน ตามหลักฟิสิกส์แล้วพลังงานไม่ได้หายไปไหน 
        แต่ถูกเปลี่ยนแปลงไปเป็นพลังงานอีกรูปแบบหนึ่ง เช่นเดียวกับแอพพลิเคชัน ความซับซ้อนในการใช้งานหากจะลดได้ 
        แม้ว่าโค้ดจะซับซ้อนมากขึ้นก็ควรจะทำ
  \item Doherty threshold -- อธิบายว่าความสะดวกสะบายจะเพิ่มขึ้นหากผู้ใช้ใช้เวลาในการรอจากขั้นตอนหนึ่งไปอีกขั้นตอนหนึ่งน้อยหรือไม่มีเลย 
        ดังนั้น ในการโหลดข้อมูลในหน้าต่างๆ ควรจัดการให้ใช้เวลาน้อยๆ เพื่อเพิ่มความต่อเนื่องในการใช้งาน
  \item Pareto's  Principle -- อธิบายว่าไม่ว่าจะเป็นเหตุการณ์ใดๆ 80 เปอร์เซ็นต์ของเหตุการณ์ที่เกิดขึ้น มาจากเพียง 20 เปอร์เซ็นต์ของสาเหตุทั้งหมดที่เป็นไปได้ 
        นั่นคือ การพัฒนารายละเอียดเล็กๆ น้อยๆ สามารถช่วยแก้ปัญหาใหญ่ๆ ได้
  \item Law of Similarity -- อธิบายว่าตามนุษย์ตัดสินว่า วัตถุที่ลักษณะเหมือนกันจะเป็นสิ่งเดียวกัน แม้ว่าวัตถุจะอยู่ห่างกันก็ตาม
        ดังนั้น ในการออกแบบส่วนประกอบต่างๆ หากส่วนประกอบใดๆ มีการทำงานที่เหมือนกัน ควรมีหน้าตาที่เหมือนกัน
  \item Law of Proximity -- อธิบายว่าวัตถุที่อยู่ข้างกันจะถูกจัดให้อยู่ในกลุ่มเดียวกัน 
        ดังนั้น ในการจัดวางส่วนประกอบต่างๆ ในหน้าหนึ่งๆ สิ่งใดที่ทำหน้าที่คล้ายคลึงกันควรอยู่ในตำแหน่งใกล้ๆ กัน
  \item Serial Position Effect -- อธิบายว่าผู้ใช้มักจดจำรายการแรกและรายการสุดท้ายของอันดับ
        ดังนั้น รายละเอียดใดที่สำคัญที่ผู้ใช้ควรทราบ ควรถูกจัดวางไว้ ณ ตำแหน่งแรกหรือตำแหน่งสุดท้าย
\end{enumerate}
% \CIreply{นำกฎเหล่านี้ไปใช้ในการออกแบบอย่างไร}

\section{ความรู้เกี่ยวกับด้านอัลกอริทึม}
\label{sec:rmp}
\subsection{Stable roommate matching problem}
Stable roommate matching problem เป็นปัญหาหนึ่งในกลุ่ม stable matching problem
โจทย์ของปัญหามีอยู่ว่า มีผู้เข้าร่วมอยู่ทั้งหมด $2n$ คนโดยทุกๆ คนจะทำการจัดอันดับความต้องการรูมเมทของตนเอง
เป็นจำนวน $2n-1$ อันดับ ซึ่งการจับคู่ผู้เข้าร่วมจะถูกนิยามเป็นคู่อันดับ $(m,n)$ ใดๆ โดยการจับคู่ $(m,n)$ ใดๆ จะเรียกว่าเป็น
stable matching ก็ต่อเมื่อ ไม่มีผู้เข้าร่วมคนอื่นคนใด หรือ $n'$ ที่พร้อมจะจับคู่กับตนเอง หรือ $m$ มากกว่า roommate $n$ ที่ได้รับ ซึ่ง $(m, n')$
จะเรียกว่าเป็นคู่อุปสรรค (blocking pair หรือ rogue couple)ให้ไม่มีการเกิด stable matching เกิดขึ้น

ซึ่งจากปัญหาข้างต้นนั้นจะเห็นได้ว่ามีลักษณะคล้ายคลึงกับโครงงานฉบับนี้และ ณ ขณะที่ศึกษา
โครงงานอยู่นั้นก็มีอัลกอริทึมที่สามารถแก้ปัญหานี้ได้แล้วนั่นคือ Irving's algorithm~\cite{irving1985efficient} ที่สามารถหาคำตอบได้ภายในเวลา $O(n^2)$

\subsection{NP-Problem}
ต่อไปนี้จะเป็นการอธิบายนิยามสั้นๆ ของกลุ่มปัญหา NP~\cite{np} เพื่อต่อยอดความรู้สู่ส่วนถัดไป
\begin{enumerate}
  \item NP-problem คือเซตของปัญหาที่สามารถหาคำตอบได้ด้วย nondeterministic 
    Turing machine (NTM) ภายในเวลา polynomial time ตามขนาดของ input
  \item  NP-hard คือปัญหาที่สามารถนำวิธีแก้ปัญหานั้นๆ ไปช่วยแก้ปัญหา NP อื่นๆ ได้
  \item  NP-complete คือปัญหาที่สามารถพิสูจน์ได้ว่าเป็นทั้ง NP-problem และ NP-hard
\end{enumerate}
ซึ่งเมื่อใดก็ตามที่ปัญหาหนึ่งๆ ถูกสรุปว่าเป็น NP-complete แล้วจะไม่นิยมหาวิธีที่ดีที่สุด
แต่จะเปลี่ยนเป็นหาวิธีประมาณการแทน เนื่องจากปัญหา NP-complete 
เป็นปัญหา NP ด้วย\CI{จึงจะต้องใช้เวลาในการหาคำตอบที่ดีที่สุดเป็น polynomial time}{really?} หรือ $O(n^k)$
แต่ $k$ สูงมากๆ จึงเรียกว่า superpolynomial ทำให้ใช้เวลานานในการหาคำตอบซึ่งไม่คุ้มกับเวลาที่เสียไป

\subsection{Stable roommates problem with triple rooms}
หากต้องการแก้ปัญหา stable roommate matching กับห้องที่มีมากกว่า 2 คน จะทำให้เกิดปัญหา
stable roommate matching with triple rooms ซึ่งเป็นปัญหา NP-complete~\cite{iwama2007stable}
ทำให้ใช้เวลานานในการหาคำตอบ ผนวกกับหอพักของมหาวิทยาลัยเชียงใหม่มีห้องที่อยู่ร่วมกัน 3 คนและ 4 คน
ทางผู้จัดทำจึงเลือกที่จะไม่หาวิธีที่ดีที่สุด ในการแก้ปัญหานี้
% งานวิจัย stable roommates problem with triple rooms (Kazuo, Shuichi และ Kazuya)~\cite{iwama2007stable}
% ได้ทำการพิสูจน์ว่า stable roommate matching กับห้องที่มี 3 คนนั้นเป็นปัญหา NP-complete 

% ตัวอย่างโจทย์ปัญหา stable roommate matching with triple rooms

\section{Containerization}
ส่วนต่อไปจากนี้จะเกี่ยวกับการ deploy เว็บแอพพลิเคชัน โดยเริ่มจากทำความรู้จักกับ 
containerization ว่าจริงๆ แล้วคืออะไร

Containerization~\cite{ctnrh} คือการนำแอพพลิเคชันที่พัฒนา, library หรือสิ่งต่างๆ ที่จำเป็นในการทำงานของแอพพลิเคชันทั้งหมด บรรจุลงในกล่องกล่องเดียว โดยเรียกกล่องดังกล่าวว่า container 
เมื่อต้องการที่จะนำแอพพลิเคชันนั้นไปทดสอบในระบบต่างๆ เพียงเปิดกล่องดังกล่าวขึ้นมา
ก็จะมีอุปกรณ์ทุกอย่างที่พร้อมจะทำให้แอพพลิเคชันทำงานได้ปกติ การทำ containerization 
จึงมีความคล่องตัวในการ deploy แอพพลิเคชันในสภาพแวดล้อมต่างๆ ได้ดี 

\section{\ifenglish%
\ifcpe CPE \else ISNE \fi knowledge used, applied, or integrated in this project
\else%
ความรู้ตามหลักสูตรซึ่งถูกนำมาใช้หรือบูรณาการในโครงงาน
\fi
}
\begin{enumerate}
  \item Database design: ทำให้มีความรู้พื้นฐานในการออกแบบและพัฒนาฐานข้อมูล
  % \item Object Oriented Program(OOP): ช่วยให้เข้าใจการออกแบบและหลักการของการเขียนโปรแกรม
  % เชิงวัตถุ
  \item Data Structures and Algorithms: นำความรู้ไปต่อยอดในการศึกษาและพัฒนาอัลกอริทึม
\end{enumerate}


\section{\ifenglish%
Extracurricular knowledge used, applied, or integrated in this project
\else%
ความรู้นอกหลักสูตรซึ่งถูกนำมาใช้หรือบูรณาการในโครงงาน
\fi
}

อธิบายถึงความรู้ต่างๆ ที่เรียนรู้ด้วยตนเอง และแนวทางการนำความรู้เหล่านั้นมาใช้ในโครงงาน
