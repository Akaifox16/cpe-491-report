\chapter{\ifenglish Background Knowledge and Theory\else ทฤษฎีที่เกี่ยวข้อง\fi}

การทำโครงงาน เริ่มต้นด้วยการศึกษาค้นคว้า ทฤษฎีที่เกี่ยวข้อง หรือ งานวิจัย/โครงงาน 
ที่เคยมีผู้นำเสนอไว้แล้ว ซึ่งเนื้อหาในบทนี้ก็จะเกี่ยวกับการอธิบายถึงสิ่งที่เกี่ยวข้องกับโครงงาน 
เพื่อให้ผู้อ่านเข้าใจเนื้อหาในบทถัดๆ ไปได้ง่ายขึ้น

\section{UX design}
\subsection{User experience (UX)}
User experience (UX) คือความสะดวกและประสบการณ์การใช้งานของผู้ใช้ระบบ 
หรือแอพพลิเคชันนั้นๆ ว่าใช้งานง่ายมากน้อยเพียงใด ความสะดวก จำนวนครั้งที่ต้องมีปฏิสัมพันธ์กับระบบ
\CI{ทิศทางการไหล}{?}ของสิ่งที่ต้องไปปฏิสัมพันธ์ด้วย

\subsection{UX laws}
\label{subsec:uxlaws}
กฏในการออกแบบ UX~\cite{uxui} มี 11 ข้อดังนี้
\begin{enumerate}
  \item Aesthetic-Usability Effect -- อธิบายว่าผู้ใช้มักมองว่าความสวยงามน่าใช้คือความใช้ง่าย 
      ซึ่งแท้จริงแล้วอาจจะแอพพลิเคชันนั้นๆ อาจจะไม่ได้ใช้งานง่ายที่สุดก็ตาม 
  \item Jakob's Law -- อธิบายว่าผู้ใช้มักคาดหวังให้เว็บที่พัฒนาใหม่นั้นทำงานเช่นเดียวกันกับเว็บที่เขาเคยมีประสบการณ์ใช้งาน
  \item Fitts's Law -- อธิบายว่าเวลาที่ผู้ใช้จะต้องใช้เพื่อปฏิสัมพันธ์กับฟังก์ชันการทำงาน จะขึ้นกับความใหญ่และระยะทางที่ต้องใช้เพื่อไปถึงเป้าหมาย
  \item Hick's Law -- อธิบายว่าเวลาที่ผู้ใช้จะต้องใช้เพื่อตัดสินใจเลือกบางอย่าง จะขึ้นอยู่กับจำนวนของตัวเลือกและความซับซ้อนของตัวเลือกนั้นๆ
  \item Miller's Law -- อธิบายว่าผู้ใช้สามารถจดจำทุกสิ่งทุกอย่างได้โดยเฉลี่ย $7\pm2$ อย่าง 
        เช่น เบอร์โทรหากเขียนว่า 09185XXXXX จะจำได้ยากกว่า 091-85X-XXXX เป็นต้น
  \item Tesler's Law -- เปรียบเทียบความซับซ้อนเป็นพลังงาน ตามหลักฟิสิกส์แล้วพลังงานไม่ได้หายไปไหน 
        แต่ถูกเปลี่ยนแปลงไปเป็นพลังงานอีกรูปแบบหนึ่ง เช่นเดียวกับแอพพลิเคชัน ความซับซ้อนในการใช้งานหากจะลดได้ 
        ต้องแปรเปลี่ยนให้เป็นความซับซ้อนของโค้ดที่ใช้พัฒนา
  \item Doherty threshold -- อธิบายว่าความสะดวกสะบายจะเพิ่มขึ้นหากผู้ใช้ใช้เวลาในการรอจากขั้นตอนหนึ่งไปอีกขั้นตอนหนึ่งน้อยหรือไม่มีเลย
  \item Pareto's  Principle -- อธิบายว่าไม่ว่าจะเป็นเหตุการณ์ใดๆ \CI{80 เปอร์เซ็นต์}{ของ?}เกิดขึ้นจาก 20 เปอร์เซ็นต์ของสาเหตุ นั่นคือการพัฒนารายละเอียดเล็กๆ น้อยๆ สามารถช่วยแก้ปัญหาใหญ่ๆ ได้
  \item Law of Similarity -- อธิบายว่าตามนุษย์ตัดสินว่า วัตถุที่ลักษณะเหมือนกันจะเป็นสิ่งเดียวกัน แม้ว่าวัตถุจะอยู่ห่างกันก็ตาม
  \item Law of Proximity -- อธิบายว่าวัตถุที่อยู่ข้างกันจะถูกจัดให้อยู่ในกลุ่มเดียวกัน
  \item Serial Position Effect -- อธิบายว่าผู้ใช้มักจดจำรายการแรกและรายการสุดท้ายของอันดับ
\end{enumerate}
\CIreply{นำกฎเหล่านี้ไปใช้ในการออกแบบอย่างไร}

\section{ความรู้เกี่ยวกับด้านอัลกอริทึม}
\label{sec:rmp}
\subsection{Stable roommate matching problem}
Stable roommate matching problem เป็นปัญหาหนึ่งในกลุ่ม stable matching problem
โจทย์ของปัญหามีอยู่ว่า มีผู้เข้าร่วมอยู่ทั้งหมด $2n$ คนโดยทุกๆ คนจะทำการจัดอันดับความต้องการรูมเมทของตนเอง
เป็นจำนวน $2n-1$ อันดับ ซึ่งการจับคู่ผู้เข้าร่วมจะถูกนิยามเป็นคู่อันดับ $(m,n)$ ใดๆ โดยการจับคู่ $(m,n)$ ใดๆ จะเรียกว่าเป็น
stable matching ก็ต่อเมื่อ ไม่มีผู้เข้าร่วมคนอื่นคนใด หรือ $n'$ ที่พร้อมจะจับคู่กับตนเอง หรือ $m$ มากกว่า roommate $n$ ที่ได้รับ ซึ่ง $(m, n')$
จะเรียกว่าเป็นคู่อุปสรรค (blocking pair หรือ rogue couple)ให้ไม่มีการเกิด stable matching เกิดขึ้น

ซึ่งจากปัญหาข้างต้นนั้นจะเห็นได้ว่ามีลักษณะคล้ายคลึงกับโครงงานฉบับนี้และ ณ ขณะที่ศึกษา
โครงงานอยู่นั้นก็มีอัลกอริทึมที่สามารถแก้ปัญหานี้ได้แล้วนั่นคือ Irving's algorithm~\cite{irving1985efficient} ที่สามารถหาคำตอบได้ภายในเวลา $O(n^2)$

\subsection{NP-Problem}
ต่อไปนี้จะเป็นการอธิบายนิยามสั้นๆ ของกลุ่มปัญหา NP~\cite{np} เพื่อต่อยอดความรู้สู่ส่วนถัดไป
\begin{enumerate}
  \item NP-problem คือเซตของปัญหาที่สามารถหาคำตอบได้ด้วย nondeterministic 
    Turing machine ภายในเวลา polynomial time ตามขนาดของ input
  \item  NP-hard คือปัญหาที่สามารถนำวิธีแก้ปัญหานั้นๆ ไปช่วยแก้ปัญหา NP อื่นๆ ได้
  \item  NP-complete คือปัญหาที่สามารถพิสูจน์ได้ว่าเป็นทั้ง NP-problem และ NP-hard
\end{enumerate}
ซึ่งเมื่อใดก็ตามที่ปัญหาหนึ่งๆ ถูกสรุปว่าเป็น NP-complete แล้วจะไม่นิยมหาวิธีที่ดีที่สุด
แต่จะเปลี่ยนเป็นหาวิธีประมาณการแทน เนื่องจากวิธีที่ดีที่สุดนั้น \CI{จะต้องใช้เวลาในการหา polynomial time}{really?} $O(n^k)$
โดยที่ $k$ สูงมากๆ เรียกว่า superpolynomial ทำให้ใช้เวลานานในการหาคำตอบซึ่งไม่คุ้มกับเวลาที่เสียไป

\subsection{Stable roommates problem with triple rooms}
งานวิจัย stable roommates problem with triple rooms (Kazuo, Shuichi และ Kazuya)~\cite{iwama2007stable}
ได้ทำการพิสูจน์ว่า stable roommate matching กับห้องที่มี 3 คนนั้นเป็นปัญหา NP-complete 
\CI{โดยยกตัวอย่างการแปลงปัญหา stable roommate matching เป็นปัญหา Partitions into Triangles 
ซึ่งเป็นปัญหา NP-complete}{really?}

% ตัวอย่างโจทย์ปัญหา stable roommate matching with triple rooms


จากงานวิจัยข้างต้น จึงสรุปได้ว่า stable roommate problem ที่ 1 ห้องมีมากกว่า 2 คนนั้น
เป็นปัญหา NP-complete ผนวกกับหอพักของมหาวิทยาลัยเชียงใหม่มีห้องที่อยู่ร่วมกัน 3 คนและ 4 คน
ทางผู้จัดทำจึงเลือกที่จะไม่หาวิธีที่ดีที่สุด ในการแก้ปัญหานี้

\section{Containerization}
ส่วนต่อไปจากนี้จะเกี่ยวกับการ deploy เว็บแอพพลิเคชัน โดยเริ่มจากทำความรู้จักกับ 
containerization ว่าจริงๆ แล้วคืออะไร

Containerization~\cite{ctnrh} คือการนำแอพพลิเคชันที่พัฒนา, library หรือสิ่งต่างๆ ที่จำเป็นในการทำงานของแอพพลิเคชันทั้งหมด บรรจุลงในกล่องกล่องเดียว โดยเรียกกล่องดังกล่าวว่า container 
เมื่อต้องการที่จะนำแอพพลิเคชันนั้นไปทดสอบในระบบต่างๆ เพียงเปิดกล่องดังกล่าวขึ้นมา
ก็จะมีอุปกรณ์ทุกอย่างที่พร้อมจะทำให้แอพพลิเคชันทำงานได้ปกติ การทำ containerization 
จึงมีความคล่องตัวในการ deploy แอพพลิเคชันในสภาพแวดล้อมต่างๆ ได้ดี 
% \section{Third section}
% Section 3 text. The dielectric constant\index{dielectric constant}
% at the air-metal interface determines
% the resonance shift\index{resonance shift} as absorption or capture occurs
% is shown in Equation~\eqref{eq:dielectric}:

% \begin{equation}\label{eq:dielectric}
% k_1=\frac{\omega}{c({1/\varepsilon_m + 1/\varepsilon_i})^{1/2}}=k_2=\frac{\omega
% \sin(\theta)\varepsilon_\mathit{air}^{1/2}}{c}
% \end{equation}

% \noindent
% where $\omega$ is the frequency of the plasmon, $c$ is the speed of
% light, $\varepsilon_m$ is the dielectric constant of the metal,
% $\varepsilon_i$ is the dielectric constant of neighboring insulator,
% and $\varepsilon_\mathit{air}$ is the dielectric constant of air.

% \section{About using figures in your report}

% define a command that produces some filler text, the lorem ipsum.
% \newcommand{\loremipsum}{
%   \textit{Lorem ipsum dolor sit amet, consectetur adipisicing elit, sed do
%   eiusmod tempor incididunt ut labore et dolore magna aliqua. Ut enim ad
%   minim veniam, quis nostrud exercitation ullamco laboris nisi ut
%   aliquip ex ea commodo consequat. Duis aute irure dolor in
%   reprehenderit in voluptate velit esse cillum dolore eu fugiat nulla
%   pariatur. Excepteur sint occaecat cupidatat non proident, sunt in
%   culpa qui officia deserunt mollit anim id est laborum.}\par}

% \begin{figure}
%   \centering

%   \fbox{
%      \parbox{.6\textwidth}{\loremipsum}
%   }

%   % To include an image in the figure, say myimage.pdf, you could use
%   % the following code. Look up the documentation for the package
%   % graphicx for more information.
%   % \includegraphics[width=\textwidth]{myimage}

%   \caption[Sample figure]{This figure is a sample containing \gls{lorem ipsum},
%   showing you how you can include figures and glossary in your report.
%   You can specify a shorter caption that will appear in the List of Figures.}
%   \label{fig:sample-figure}
% \end{figure}

% Using \verb.\label. and \verb.\ref. commands allows us to refer to
% figures easily. If we can refer to Figures
% \ref{fig:walrus} and \ref{fig:sample-figure} by name in the {\LaTeX}
% source code, then we will not need to update the code that refers to it
% even if the placement or ordering of the figures changes.

% \loremipsum\loremipsum

% This code demonstrates how to get a landscape table or figure. It
% uses the package lscape to turn everything but the page number into
% landscape orientation. Everything should be included within an
% \afterpage{ .... } to avoid causing a page break too early.
% \afterpage{
%   \begin{landscape}
%   \begin{table}
%     \caption{Sample landscape table}
%     \label{tab:sample-table}

%     \centering

%     \begin{tabular}{c||c|c}
%         Year & A & B \\
%         \hline\hline
%         1989 & 12 & 23 \\
%         1990 & 4 & 9 \\
%         1991 & 3 & 6 \\
%     \end{tabular}
%   \end{table}
%   \end{landscape}
% }

% \loremipsum\loremipsum\loremipsum

% \section{Overfull hbox}

% When the \verb.semifinal. option is passed to the \verb.cpecmu. document class,
% any line that is longer than the line width, i.e., an overfull hbox, will be
% highlighted with a black solid rule:
% \begin{center}
% \begin{minipage}{2em}
% juxtaposition
% \end{minipage}
% \end{center}

\section{\ifenglish%
\ifcpe CPE \else ISNE \fi knowledge used, applied, or integrated in this project
\else%
ความรู้ตามหลักสูตรซึ่งถูกนำมาใช้หรือบูรณาการในโครงงาน
\fi
}
\begin{enumerate}
  \item Database design: ทำให้มีความรู้พื้นฐานในการออกแบบและพัฒนาฐานข้อมูล
  % \item Object Oriented Program(OOP): ช่วยให้เข้าใจการออกแบบและหลักการของการเขียนโปรแกรม
  % เชิงวัตถุ
  \item Data Structures and Algorithms: นำความรู้ไปต่อยอดในการศึกษาและพัฒนาอัลกอริทึม
\end{enumerate}


\section{\ifenglish%
Extracurricular knowledge used, applied, or integrated in this project
\else%
ความรู้นอกหลักสูตรซึ่งถูกนำมาใช้หรือบูรณาการในโครงงาน
\fi
}

อธิบายถึงความรู้ต่างๆ ที่เรียนรู้ด้วยตนเอง และแนวทางการนำความรู้เหล่านั้นมาใช้ในโครงงาน
