\chapter{\ifenglish Background Knowledge and Theory\else ทฤษฎีที่เกี่ยวข้อง\fi}

การทำโครงงาน เริ่มต้นด้วยการศึกษาค้นคว้า ทฤษฎีที่เกี่ยวข้อง หรือ งานวิจัย/โครงงาน 
ที่เคยมีผู้นำเสนอไว้แล้ว ซึ่งเนื้อหาในบทนี้ก็จะเกี่ยวกับการอธิบายถึงสิ่งที่เกี่ยวข้องกับโครงงาน 
เพื่อให้ผู้อ่านเข้าใจเนื้อหาในบทถัดๆ ไปได้ง่ายขึ้น

\section{งานวิจัยที่คล้ายคลึงกัน}
\label{sec:rmp}
\subsection{Stable roommate matching problem}
Stable roommate matching problem เป็นปัญหาหนึ่งในกลุ่ม stable matching problem
โจทย์ของปัญหามีอยู่ว่า มีผู้เข้าร่วมอยู่ทั้งหมด $2n$ คนโดยทุกๆ คนจะทำการจัดอันดับความต้องการรูมเมทของตนเอง
เป็นจำนวน $2n-1$ อันดับ ซึ่งการจับคู่ผู้เข้าร่วมจะถูกนิยามเป็นคู่อันดับ $(m,n)$ ใดๆ โดยการจับคู่ $(m,n)$ ใดๆ จะเรียกว่าเป็น
stable matching ก็ต่อเมื่อ ไม่มีผู้เข้าร่วมคนอื่นคนใดนิยามว่า $n'$ กับ $m$ใดๆ ต้องการที่จะจับคู่กันมากกว่าคู่ที่ตนเองได้รับ ซึ่ง $(m, n')$
จะเรียกว่าเป็นคู่อุปสรรค(blocking pair หรือ rogue couple) ให้ไม่มี stable matching เกิดขึ้น

ซึ่งจากปัญหาข้างต้นนั้นจะเห็นได้ว่ามีลักษณะคล้ายคลึงกับโครงงานฉบับนี้และ ณ ขณะที่ศึกษา
โครงงานอยู่นั้นได้มีอัลกอริทึมที่สามารถแก้ปัญหานี้ได้แล้ว นั่นคือ Irving's algorithm
~\cite{irving1985efficient} ที่สามารถหาคำตอบได้ภายในเวลา $O(n^2)$

\subsection{NP-Problem}
ต่อไปนี้จะเป็นการอธิบายนิยามสั้นๆ ของกลุ่มปัญหา NP~\cite{np} เพื่อต่อยอดความรู้สู่ส่วนถัดไป
\begin{enumerate}
      \item NP-problem คือเซตของปัญหาที่สามารถหาคำตอบได้ด้วย Nondeterministic
            Turing Machine (NTM) ภายในเวลา polynomial time ตามขนาดของ input
      \item  NP-hard คือปัญหาที่สามารถนำวิธีแก้ปัญหานั้นๆ ไปช่วยแก้ปัญหา NP อื่นๆ ได้
      \item  NP-complete คือปัญหาที่สามารถพิสูจน์ได้ว่าเป็นทั้ง NP-problem และ NP-hard
\end{enumerate}
ซึ่งเมื่อใดก็ตามที่ปัญหาหนึ่งๆ ถูกสรุปว่าเป็น NP-complete ผู้แก้ปัญหามักจะไม่พยายามหาวิธีที่ดีที่สุด
แต่จะเปลี่ยนเป็นหาวิธีประมาณการแทน เนื่องจากปัญหา NP-complete 
ต้องใช้เวลาในการหาคำตอบที่ดีที่สุดเป็น super-polynomial~\cite{np}
ซึ่งเป็นเวลาที่นานไม่คุ้มกับเวลาที่เสียไป



\subsection{Stable roommates problem with triple rooms}
เนื่องจากทางหอพักของมหาวิทยาลัยเชียงใหม่มีห้องที่อยู่ร่วมกันมากกว่า 2 คน หากต้องการแก้ปัญหา stable roommate matching 
กับห้องที่มี 3 คน จะทำให้กลายเป็นstable roommate matching with triple rooms ซึ่งเป็นปัญหา 
NP-complete~\cite{iwama2007stable}ทำให้ใช้เวลานานในการหาคำตอบ ทางผู้จัดทำจึงเลือกที่จะไม่หาวิธีที่ดีที่สุด ในการแก้ปัญหานี้
แต่จะใช้ greedy algorithm แทนโดยจะขยายความการทำงานในบทที่ 3

\section{โครงสร้างเว็บแอพพลิเคชัน}
\subsection{เว็บแอพพลิเคชัน}
\subsection{แอพพลิเคชันแบบ Tree-tier}
\subsection{โครงสร้าง Microservice}

\section{Technology}
\subsection{Front-end}
\subsubsection{Next.js}
\subsubsection{Typescript}
\subsubsection{tRPC}
\subsubsection{RPC}
\subsection{Back-end}
\subsubsection{Go}
\subsubsection{Concurrency}
\subsection{Deployment}
ส่วนต่อไปจากนี้จะเกี่ยวกับการ deploy เว็บแอพพลิเคชัน โดยเริ่มจากทำความรู้จักกับ 
containerization ว่าจริงๆ แล้วคืออะไร
\subsubsection{Containerization}
Containerization~\cite{ctnrh} คือการนำแอพพลิเคชันที่พัฒนา, library หรือสิ่งต่างๆ 
ที่จำเป็นในการทำงานของแอพพลิเคชันทั้งหมด บรรจุลงในกล่องกล่องเดียว โดยเรียกกล่องดังกล่าวว่า container 
เมื่อต้องการที่จะนำแอพพลิเคชันนั้นไปทดสอบในระบบต่างๆ เพียงเปิดกล่องดังกล่าวขึ้นมา
ก็จะมีอุปกรณ์ทุกอย่างที่พร้อมจะทำให้แอพพลิเคชันทำงานได้ปกติ การทำ containerization 
จึงมีความคล่องตัวในการ deploy แอพพลิเคชันในสภาพแวดล้อมต่างๆ ได้ดี 
\subsubsection{Docker}


\section{\ifenglish%
        \ifcpe CPE \else ISNE \fi knowledge used, applied, or integrated in this project
  \else%
        ความรู้ตามหลักสูตรซึ่งถูกนำมาใช้หรือบูรณาการในโครงงาน
  \fi
 }
\begin{enumerate}
      \item Database design: ทำให้มีความรู้พื้นฐานในการออกแบบและพัฒนาฐานข้อมูล
      \item Data Structures and Algorithms: นำความรู้ไปต่อยอดในการศึกษาและพัฒนาอัลกอริทึม
      \item Basic computer engineer laboratory: นำความรู้ไปต่อยอดการพัฒนาเว็บแอพพลิเคชัน
      \item Software Engineering: ช่วยในการวางแผน และระเบียบในการพัฒนา
\end{enumerate}


\section{\ifenglish%
        Extracurricular knowledge used, applied, or integrated in this project
  \else%
        ความรู้นอกหลักสูตรซึ่งถูกนำมาใช้หรือบูรณาการในโครงงาน
  \fi
 }
\begin{enumerate}
      \item Containerization: นำความรู้มาใช้ในการ deploy เว็บแอพพลิเคชัน
      \item Infrastructure as Code: ใช้ในการตั้งค่าฐานข้อมูลแบบอัตโนมัติ
      \item RPC: ใช้ในการเชื่อมต่อระบบฝั่ง Front-end และ Back-end เข้าด้วยกัน
\end{enumerate}
