\chapter{\ifenglish Conclusions and Discussions\else บทสรุปและข้อเสนอแนะ\fi}

\section{\ifenglish Conclusions\else สรุปผล\fi}

นศ. ควรสรุปถึงข้อจำกัดของระบบในด้านต่างๆ ที่ระบบมีในเนื้อหาส่วนนี้ด้วย

\section{\ifenglish Challenges\else ปัญหาที่พบและแนวทางการแก้ไข\fi}

ในการทำโครงงานนี้ พบว่าเกิดปัญหาหลักๆ ดังนี้
\begin{enumerate}
  \item การจับคู่นั้นจริงๆ แล้วมีอัลกอริทึมที่ต้องคิด และพัฒนาทั้งหมด 3 ส่วน ซึ่งต้องใช้ข้อมูลจำนวนมากประกอบการพัฒนา
        ซึ่งทำให้ภาระงานเพิ่มขึ้นมากจากที่คาดการณ์เอาไว้ทำให้ไม่เหลือเวลาในการพัฒนาฟีเจอร์ส่วนอื่นๆ เช่น หน้าโปรไฟล์ หรือหน้าคู่มือการใช้งานที่ออกแบบไว้เป็นคู่มือที่ผู้ใช้จะสามารถปฏิสัมพันธ์ด้วยได้
  \item tRPC เป็นไลบรารีที่ค่อนข้างใหม่ ซึ่งทำให้การทดสอบและ การแก้ปัญหาดำเนินไปด้วยความยากลำบาก
\end{enumerate}


\section{\ifenglish%
Suggestions and further improvements
\else%
ข้อเสนอแนะและแนวทางการพัฒนาต่อ
\fi
}
% \begin{enumerate}
%   \item การจัดการ
% \end{enumerate}
เนื่องจากโครงการนี้ประกอบด้วยหลายฟีเจอร์ที่ใหญ่ ซึ่งจะทำให้มีงานย่อยๆที่ต้องทำในแต่ละขั้นตอนเยอะจนอาจจะติดตามความคืบหน้าได้ลำบาก
จึงอยากแนะนำให้ลองใช้ซอฟต์แวร์จัดการโปรเจคเช่น Jira, Notion หรือ Trello

เนื่องจากเครื่องมือทดสอบ API เช่น Postman นั้นยังไม่รองรับกับ tRPC จึงอยากแนะนำให้ใช้ tRPC เป็นเพียงเหมือนสะพานเชื่อมระหว่าง Front-end
และ Back-end เท่านั้น ให้พัฒนา Back-end ด้วยไลบรารีหรือเฟรมเวิร์คเช่น Express หรือ NestJS เพื่อให้ง่ายต่อการทดสอบ

ในส่วนของการจับคู่นั้น วิธีการที่ออกแบบไว้ยังเป็นการไล่จับคู่ทีละคู่ๆ ซึ่งอาจจะหาวิธีการที่จะสามารถช่วยให้สามารถลดเวลาในการประมวลผลลงได้ 
จะทำให้แอพพลิเคชันนี้เหมาะสมกับการนำไปใช้งานมากยิ่งขึ้น เช่นการแบ่งกลุ่มของผู้ใช้ในการเลือกจับกลุ่ม อาจจะเป็นอัลกอริทึมที่มีลักษณะคล้ายๆกับการหา 
K-Nearest-Neigbour

เพิ่มส่วนการทำงานของฝั่งผู้ดูแล เช่น การตั้งค่าเวลาเปิด-ปิดระบบ การจัดการห้องที่สามารถ
