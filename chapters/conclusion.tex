\chapter{\ifenglish Conclusions and Discussions\else บทสรุปและข้อเสนอแนะ\fi}

\section{\ifenglish Conclusions\else สรุปผล\fi}

% นศ. ควรสรุปถึงข้อจำกัดของระบบในด้านต่างๆ ที่ระบบมีในเนื้อหาส่วนนี้ด้วย
แอพพลิเคชันมีศักยภาพในการใช้งานจริงในด้านการออกแบบ เพียงแต่แอพพลิเคชันนั้นยังไม่สามารถใช้ได้เพราะหลาย ๆ ฟีเจอร์ยังพัฒนาไม่แล้วเสร็จดี
ในส่วนของอัลกอริทึมนั้นยังสามารถพัฒนาได้อีกเพิ่มเติมหากจะนำไปใช้งานจริง เพราะยังขาดส่วนของความเร็วเพราะปัญหาในการจับคู่รูมเมทที่ห้อง 3 คนนั้นเป็น
NP-Complete 

\section{\ifenglish Challenges\else ปัญหาที่พบและแนวทางการแก้ไข\fi}

ในการทำโครงงานนี้ พบว่าเกิดปัญหาหลักๆ ดังนี้
\begin{enumerate}
  \item เนื่องจากโครงการนี้ประกอบด้วยหลายฟีเจอร์ที่ใหญ่ ซึ่งจะทำให้มีงานย่อยๆที่ต้องทำในแต่ละขั้นตอนเยอะจนอาจจะติดตามความคืบหน้าได้ลำบาก
        จึงอยากแนะนำให้ลองใช้ซอฟต์แวร์จัดการโปรเจคเช่น Jira, Notion หรือ Trello
  \item การจับคู่นั้นจริงๆ แล้วมีอัลกอริทึมที่ต้องคิด และพัฒนาทั้งหมด 3 ส่วน ซึ่งต้องใช้ข้อมูลจำนวนมากประกอบการพัฒนา
        ซึ่งทำให้ภาระงานเพิ่มขึ้นมากจากที่คาดการณ์เอาไว้ทำให้ไม่เหลือเวลาในการพัฒนาฟีเจอร์ส่วนอื่นๆ เช่น หน้าโปรไฟล์ 
        หรือหน้าคู่มือการใช้งานที่ออกแบบไว้เป็นคู่มือที่ผู้ใช้จะสามารถปฏิสัมพันธ์ด้วยได้ หากโครงการที่ทำเป็นอัลกอริทึมแนะนำว่าให้ทำส่วนอัลกอริทึมให้เสร็จก่อน
        หากอัลกอริทึมเสร็จแล้วค่อยวางแผนทำแอพพลิเคชัน
  \item tRPC เป็นไลบรารีที่ค่อนข้างใหม่ ซึ่งทำให้การทดสอบและ การแก้ปัญหาดำเนินไปด้วยความยากลำบาก ต้องแยกส่วน backend มาใช้เป็น 
        NestJS หรือ Express หรือเครื่องมืออื่นๆ แล้วใช้ tRPC ในการเชื่อมส่วนของการตรวจสอบประเภทข้อมูลระหว่าง frontend กับ backend
\end{enumerate}


\section{\ifenglish%
Suggestions and further improvements
\else%
ข้อเสนอแนะและแนวทางการพัฒนาต่อ
\fi
}
\begin{enumerate}
  \item ในส่วนของการจับคู่นั้น วิธีการที่ออกแบบไว้ยังเป็นการไล่จับคู่ทีละคู่ๆ ซึ่งอาจจะหาวิธีการที่จะสามารถช่วยให้สามารถลดเวลาในการประมวลผลลงได้ 
  จะทำให้แอพพลิเคชันนี้เหมาะสมกับการนำไปใช้งานมากยิ่งขึ้น เช่นการแบ่งกลุ่มของผู้ใช้ในการเลือกจับกลุ่ม อาจจะเป็นอัลกอริทึมที่มีการยึดเอาแนวคิดคล้ายกับการแก้ปัญหา 
  K-Nearest-Neigbour 
  \item ในหน้า Summary เพิ่มการปรับลำดับของคุณลักษณะที่แสดงเรียงตามลำดับน้ำหนัก เพื่อให้ผู้ใช้รู้ว่าจริงๆ แล้วตนเองสนใจที่คุณลักษณะใดบ้าง
  \item ปรับปรุงอินพุตที่ใช้กรอกเวลาที่ต้องการความสงบให้ใช้งานได้ง่ายมากยิ่งขึ้น ลดจำนวนการคลิก อาจทำเป็นนาฬิกาแล้วมีช่วงให้ขยับตามความต้องการ
  \item แอพพลิเคชันปัจจุบันยังไม่ใช่แอพพลิเคชันที่เป็น Responsive Design ให้เหมาะกับการใช้งานของผู้ใช้งานโทรศัพท์
\end{enumerate}


% เนื่องจากเครื่องมือทดสอบ API เช่น Postman นั้นยังไม่รองรับกับ tRPC จึงอยากแนะนำให้ใช้ tRPC เป็นเพียงเหมือนสะพานเชื่อมระหว่าง Front-end
% และ Back-end เท่านั้น ให้พัฒนา Back-end ด้วยไลบรารีหรือเฟรมเวิร์คเช่น Express หรือ NestJS เพื่อให้ง่ายต่อการทดสอบ







