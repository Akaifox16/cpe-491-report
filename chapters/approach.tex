\chapter{\ifproject%
\ifenglish Project Structure and Methodology\else โครงสร้างและขั้นตอนการทำงาน\fi
\else%
\ifenglish Project Structure\else โครงสร้างของโครงงาน\fi
\fi
}

ในบทนี้จะกล่าวถึงหลักการ และการออกแบบระบบ

\makeatletter

% \renewcommand\section{\@startsection {section}{1}{\z@}%
%                                    {13.5ex \@plus -1ex \@minus -.2ex}%
%                                    {2.3ex \@plus.2ex}%
%                                    {\normalfont\large\bfseries}}

\makeatother
%\vspace{2ex}
% \titleformat{\section}{\normalfont\bfseries}{\thesection}{1em}{}
% \titlespacing*{\section}{0pt}{10ex}{0pt}
% \begin{figure}
% \begin{center}
% \includegraphics{photo/800px-Briny_Beach.jpg}
% \end{center}
% \caption[Poem]{The Walrus and the Carpenter}
% \label{fig:walrus}
% \end{figure}

\section{Software Architecture}
เว็บแอพพลิเคชันใช้ Three tier architecture ในการออกแบบระบบ โดย Front-end ใช้ Next.js 
ที่เป็น React framework ส่วนBack-end ใช้ Gin Gonic ซึ่งเป็น Go framework ติดต่อกับ Front-end 
ด้วย GraphQL API และฐานข้อมูลใช้ mySQL เป็น relational database management system(RDBMS)
\begin{figure}[h]
\begin{center}
\includegraphics[width=50mm,scale=0.5]{photo/threetierarch.png}
\end{center}
\caption{three tier architecture}
\label{fig:three-tier}
\end{figure}

\section{Feature}
\subsection{End-user}
\begin{enumerate}
  \item ระบบยืนยันตัวตน : ให้ผูู้ใช้ลงทะเบียนสมัครสมาชิกและเข้าสู่ระบบเพื่อที่จะสามารถใช้งานระบบต่างๆของระบบได้ต่อไป
  \begin{figure}[h]
  \begin{center}
  \includegraphics[width=50mm,scale=0.5]{photo/homepageNoAuth.png}
  \end{center}
  \caption{homepage ที่ผู้ใช้ยังไม่ได้เข้าสู่ระบบ}
  \label{fig:hp-no-auth}
  \end{figure}

  \begin{figure}[h]
  \begin{center}
  \includegraphics[width=50mm,scale=0.5]{photo/homepageLogin.png}
  \end{center}
  \caption{homepage ขณะกำลังจะเข้าสู่ระบบ}
  \label{fig:hp-login}
  \end{figure}

  \begin{figure}[h]
  \begin{center}
  \includegraphics[width=50mm,scale=0.5]{photo/homepageReg.png}
  \end{center}
  \caption{homepage ที่ผู้ใช้กำลังสมัครสมาชิก}
  \label{fig:hp-reg}
  \end{figure}

  \begin{figure}[h]
  \begin{center}
  \includegraphics[width=50mm,scale=0.5]{photo/homepageAuth.png}
  \end{center}
  \caption{homepage ที่ผู้ใช้เข้าสู่ระบบแล้ว}
  \label{fig:hp-auth}
  \end{figure}

  \item ระบบการจัดอันดับคุณลักษณะ : ให้ผู้ใช้ที่ลงชื่อเข้าใช้แล้วเลือกคุณสมบัติของห้องและเพื่อนร่วมห้องที่ต้องการ
  \begin{figure}[h]
  \begin{center}
  \includegraphics[width=50mm,scale=0.5]{photo/Preference.png}
  \end{center}
  \caption{หน้า preference ให้ผู้ใช้เลือกคุณลักษณะที่ต้องการ}
  \label{fig:preference}
  \end{figure}
  
  \item ระบบปรับจูนคุณลักษณะ : ระบบจะทำการนำโปรไฟล์ของผู้ใช้มาแสดงให้ผู้ใช้เลือกว่าหากเป็นเพื่อนร่วมห้องที่มี
  คุณลักษณะตามโปรไฟล์ ผู้ใช้จะเลือกเพื่อนร่วมห้องคนนั้นหรือไม่ เพื่อคำนวณหาระดับความใส่ใจในคุณลักษณะต่างๆของผู้ใช้ 
  ตัวอย่างดังรูปที่ 3.6
  \begin{figure}[h]
  \begin{center}
  \includegraphics[width=50mm,scale=0.5]{photo/finetune.png}
  \end{center}
  \caption{หน้า fine tune}
  \label{fig:finetune}
  \end{figure}

  \item ระบบรายงานสรุปผล : แสดงความคืบหน้าว่า ณ ปัจจุบันผู้ใช้มีโอกาสจะได้จับคู่กับเพื่อนร่วมห้องที่มีคุณลักษณะอย่างไร 
        และจะได้ห้องแบบใด 
  \begin{figure}[h]
  \begin{center}
  \includegraphics[width=50mm,scale=0.5]{photo/resultSelectedProfile.png}
  \end{center}
  \caption{หน้าสรุปผล 10 อันดับสูงสุดที่มีโอกาสได้จับคู่กับผู้ใช้}
  \label{fig:select-profile}
  \end{figure}
  \begin{figure}[h]
  \begin{center}
  \includegraphics[width=50mm,scale=0.5]{photo/resultPreferenceSummary.png}
  \end{center}
  \caption{หน้าสรุปผลคุณลักษณะที่ผู้ใช้ให้ความสนใจ}
  \label{fig:summary}
  \end{figure}

  \item ระบบคู่มือการใช้งาน : แสดงคู่มือการใช้งานของระบบสำหรับผู้ใช้ทั่วไป ไม่รวมผู้ใช้ที่เป็นผู้ดูแลระบบ
  \begin{figure}[h]
  \begin{center}
  \includegraphics[width=50mm,scale=0.5]{photo/Guideline.png}
  \end{center}
  \caption{หน้าคู่มือการใช้งาน}
  \label{fig:guideline}
  \end{figure}

  \item ระบบโปรไฟล์ผู้ใช้ : ให้ผู้ใช้เข้าไปปรับแต่งคุณลักษณะของตนเองเพื่อนำไปใช้ในการจับคู่กับผู้อื่น
  \begin{figure}[h]
  \begin{center}
  \includegraphics[width=50mm,scale=0.5]{photo/profile.png}
  \end{center}
  \caption{หน้าโปรไฟล์}
  \label{fig:profile}
  \end{figure}
  
\end{enumerate}
\subsection{Administrator}
\begin{enumerate}
  \item ระบบจัดการเวลาเปิด-ปิดวันลงทะเบียน : ให้ผู้ดูแลระบบสามารถตั้งค่าเวลาเปิด-ปิด
  วันลงทะเบียนของผู้ใช้ทั่วไป
  \begin{figure}[h]
  \begin{center}
  \includegraphics[width=50mm,scale=0.5]{photo/adminDashboard.png}
  \end{center}
  \caption{หน้า homepage ของผู้ดูแลระบบ}
  \label{fig:dashboard}
  \end{figure}
  \begin{figure}[h]
  \begin{center}
  \includegraphics[width=50mm,scale=0.5]{photo/datepicker.png}
  \end{center}
  \caption{ตัวอย่างการเปลี่ยนวันที่เปิด-ปิดระบบ}
  \label{fig:datepicker}
  \end{figure}
  \begin{figure}[h]
  \begin{center}
  \includegraphics[width=50mm,scale=0.5]{photo/timepicker.png}
  \end{center}
  \caption{ตัวอย่างการเปลี่ยนเวลาที่เปิด-ปิดระบบ}
  \label{fig:timepicker}
  \end{figure}
  \begin{figure}[h]
  \begin{center}
  \includegraphics[width=50mm,scale=0.5]{photo/confirmation.png}
  \end{center}
  \caption{ตัวอย่างแจ้งเตือนเพื่อยืนยันการตั้งค่า}
  \label{fig:confirm-date-time}
  \end{figure}

  \item ระบบจัดการฐานข้อมูลและการตั้งค่าหอพักที่ใช้ในการลงทะเบียน : ให้ผู้ดูแลสามารถเลือกจัดการห้องและหอที่จะเปิดให้ลงทะเบียน
  \begin{figure}[h]
  \begin{center}
  \includegraphics[width=50mm,scale=0.5]{photo/dormmgmt.png}
  \end{center}
  \caption{หน้าการจัดการหอพักของระบบจอง}
  \label{fig:mgmt}
  \end{figure}
  \begin{figure}[h]
  \begin{center}
  \includegraphics[width=50mm,scale=0.5]{photo/dormadd.png}
  \end{center}
  \caption{หน้าการเพิ่มหอพักในระบบลงทะเบียน}
  \label{fig:add-dorm}
  \end{figure}
  
  \item ระบบจัดการไฟล์ที่ใช้ในการจัดการระบบ : ให้ผู้ดูแลระบบสามารถเข้ามาดาว์นโหลดไฟล์ที่ใช้ในการจัดการระบบฐานข้อมูล
  \begin{figure}[h]
  \begin{center}
  \includegraphics[width=50mm,scale=0.5]{photo/format.png}
  \end{center}
  \caption{หน้าดาว์นโหลดไฟล์}
  \label{fig:format-db}
  \end{figure}
\end{enumerate}